%\documentclass[11pt]{report}



%\begin{document}

\chapter{Quantum oblivious linear evaluation}

Oblivious Linear Evaluation (OLE) is a cryptographic task that permits two distrustful parties, say Alice and Bob, to jointly compute the output of a linear function $f(x)=ax+b$ in some finite field, $\mathbb{F}$. Alice provides inputs $a, b\in\mathbb{F}$ and Bob provides $x\in\mathbb{F}$, while the output, $f(x)$, becomes available only to Bob. As the parties are distrustful, a secure OLE protocol should not permit Alice to learn anything about Bob's input, while also Alice's inputs should remain unknown to Bob.  OLE can be seen as a generalization of Oblivious Transfer (OT) \cite{Rabin81}, a basic primitive for secure two-party computation, which is a special case of secure multi-party computation \cite{Goldreichbook04,CCD88,Canetti00MPC}. OT has been shown to be complete  for secure multi-party computation \cite{Kilian}, i.e., any such task, including OLE, can be achieved given an OT implementation. 
A compelling reason to study OLE protocols is that they can serve as building blocks for the secure evaluation of arithmetic circuits \cite{AIK11,DKMQ12,GNN17,DGNBNT17}, just like OT allows the secure evaluation of boolean circuits \cite{GMW87}. Specifically, OLE can be used to generate multiplication triples which are the basic tool for securely computing multiplication gates \cite{DGNBNT17}. Besides that, OLE has applications in more tasks for two-party secure computation  \cite{IPS09,ADINZ17,BCGI18,HIMV19,CDIKLOV19} and  Private Set Intersection \cite{GN19}.
 
Impagliazzo and Rudich  proved that OT protocols require public-key cryptography and cannot just rely on symmetric cryptography \cite{IR89}. Consequently, OLE cannot rely on symmetric cryptography either, and we need to resort to public-key cryptography.  However, Shor's  quantum algorithm \cite{Shor94}  poses a threat to the currently deployed public-key systems, motivating the search for protocols secure against quantum attacks. Bennet et al. \cite{BBCS92} and Cr{\'e}peau \cite{C94} proposed the first protocols for Quantum OT (QOT). The no-go theorems by Lo and Chau \cite{LC97,LC98} and, independently, by Mayer \cite{Mayer97}, implied that an unconditionally secure QOT without any additional assumption is impossible. 
A decade later, Damg\r{a}rd et al. proved the security of QOT in the stand-alone model under additional assumptions \cite{DFLSS09}, and Unruh \cite{Unruh10} showed that it is statistically secure and universally composable  with access only to ideal commitments.
As far as quantum OLE (QOLE) is concerned, to the best of our knowledge, no protocol has been proposed as of now.
Analogously to the classical case it is expected that one can implement QOLE based on QOT protocols. That said, in this work we propose a protocol for QOLE that, additionally, does not rely on any QOT implementation.

OLE is commonly generalized to Vector OLE (VOLE). In this setting, Alice defines a set of $k$ linear functions $(\bm{a}, \bm{b})\in\mathbb{F}^k\times\mathbb{F}^k$ and Bob receives the evaluation of all these functions on a specified element $x\in\mathbb{F}$, i.e. $\bm{f}:=\bm{a} x+ \bm{b}$. One can think of VOLE as the arithmetic analog of string OT and show how it can be used  in certain Secure Arithmetic Computation and Non-Interactive Zero Knowledge proofs \cite{BCGI18}. Ghosh et. al  put further in evidence the usefulness of VOLE by showing that it serves as the building block of Oblivious Polynomial Evaluation \cite{GNN17}, a primitive which allows more sophisticated applications, such as password authentication, secure  list intersection,  anonymous complaint boxes \cite{NP06}, anonymous initialization for secure metering of client visits in servers \cite{NP99},  secure Taylor approximation of relevant functions (e.g. logarithm) \cite{LP02}, secure set intersection \cite{H18} and distributed generation of RSA keys \cite{G99}.  We also show how our QOLE protocol can be adapted to achieve secure VOLE.
%\bibliography{bibforthesis}
%\bibliographystyle{unsrt}
%\end{document}
