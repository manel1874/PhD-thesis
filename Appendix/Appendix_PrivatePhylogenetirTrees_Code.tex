
\appendix

\chapter{Jukes-Cantor distance for CBMC-GC}\label{appendix:appendixA}
%\addcontentsline{toc}{chapter}{Appendix A}

%\begin{appendices}
%
%\section{Jukes-Cantor code for CBMC-GC}
%\label{appendix:JC}

The boolean circuit that represents the Jukes-Cantor distance receives as inputs two four-based sequences ($A$, $C$, $G$, $T$) with size $32\, 000$. Since we are using a boolean circuit representation, the nucleotide sequences must be represented in binary. So, by convention, we use the following 2-bit encoding: $A = 00$, $C = 01$, $G = 10$ and $T = 11$. As a result, we start by defining a sequence type of size $4\, 000$ with the $\texttt{unsigned short}$ type elements (Figure~\ref{fig:jc_cbmc-gc}, lines $1-5$). In fact, the type $\texttt{Array\_Seq}$ saves $4\, 000 \times 16 = 64\, 000$ bits. Each element of $\texttt{Array\_Seq}$ represents a small sequence of eight elements. This is an implementation choice that renders a good compromise between accuracy level and circuit size.

As we saw in the main text, the hamming distance between two binary strings can then be easily computed by XORing them and counting the number of 1’s. This last operation is commonly called $\texttt{popcount}$. We cannot directly apply this approach because our sequences are in fact four-based sequences. In fact, our version of the $\texttt{popcount}$ function is only interested in computing the number of 2-bit elements that are different between both sequences.

We follow a tailored divide-and-conquer technique. The original technique is described by Henry Warren in his book “Hacker’s Delight”, chapter 5 \citep{W12}. In summary, the original technique starts by counting in parallel the number of 1’s inside each 2-bit block and saves it in 2-bit blocks. Then, it adds two 2-bit blocks and saves the result in a 4-bit block. It continues until we get the final sum. If we follow directly this approach we might run into wrong results as described in the main text. For our case, instead of counting the number of 1’s inside every 2-bit block, we only care if there is one element 1 inside each 2-bit block. This simply indicates that the elements at that site are different. This is achieved by applying an $\texttt{OR}$ operation (represented by $\texttt{|}$) to the bits inside each 2-bit block (Figure~\ref{fig:jc_cbmc-gc}, line $14$). For 4-bit blocks and above we follow the same recipe of the original divide-and-conquer technique. 

The main function that computes the Hamming weight between two nucleotide sequences $\texttt{INPUT\_A}$ and $\texttt{INPUT\_B}$ is defined by the function $\texttt{mpc\_main}$, line $21$. It outputs the inverse of the hamming weight: $\texttt{total/distance}$ line $37$. Since we know the hamming weight lies between $0$ and $1$, it renders smaller circuits to use the native integer division operator, $/$, from the CBMC-GC tool and then invert the output after the Yao computation. Otherwise, we would need a fixed precision representation to output decimal numbers.

Below we describe the variables used in the $\texttt{mpc\_main}$ function:
\begin{itemize}
\item $\texttt{INPUT\_A}$ and $\texttt{INPUT\_B}$: the binary input sequences of Alice and Bob, respectively. Following the CBMC-GC convention, the input elements must start with the identifier $\texttt{INPUT\_}$.

\item $\texttt{OUTPUT\_distance}$: the inverse of the hamming weight. Following the CBMC-GC convention, the output element must start with the identifier $\texttt{OUTPUT\_}$.

\item $\texttt{total}$: keeps track of the number of elements that can be compared between aligned sequences.

\item $\texttt{distance}$: keeps track of the hamming distance between both sequences.

\item $\texttt{count\_axorb}$: saves the number of elements that are different in a 16-bit block sequence (i.e. in $\texttt{INPUT\_A.el[i]\^{}INPUT\_B.el[i]}$).
\end{itemize}


\begin{figure}
\begin{lstlisting}[language=C]
#define LEN_SEQ 4000

typedef struct {
	unsigned short el[LEN_SEQ];
} Array_Seq;

const unsigned int m1  = 0x55555555; //binary: 0101...
const unsigned int m2  = 0x33333333; //binary: 00110011..
const unsigned int m4  = 0x0f0f0f0f; //binary: 4 zeros, 4 ones ...
const unsigned int m8  = 0x00ff00ff; //binary: 8 zeros, 8 ones ...

unsigned int popcount(unsigned short INPUT_B_x) {
    unsigned int x = INPUT_B_x;
    x = (x & m1 ) | ((x >>  1) & m1 ); // changed step
    x = (x & m2 ) + ((x >>  2) & m2 );
    x = (x & m4 ) + ((x >>  4) & m4 ); 
    x = (x & m8 ) + ((x >>  8) & m8 );
    return x;
}

void mpc_main(Array_Seq INPUT_A, Array_Seq INPUT_B){
	unsigned int distance = 0;
	int total = 0;
	for(int i=0; i<LEN_SEQ; i++){		
		int count_a = popcount(INPUT_A.el[i]);
		int count_b = popcount(INPUT_B.el[i]);
		if(count_a > 0 && count_b > 0){
			int count_axorb = popcount(INPUT_A.el[i]^INPUT_B.el[i]);			
			if(count_axorb == 1){
				distance = distance + 1;
			}
			total = total + 8;
		}
	}
	unsigned int OUTPUT_distance;
	if(distance > 0){
		OUTPUT_distance = total/distance;
	} else {
		OUTPUT_distance = 0;
	}
}
\end{lstlisting}
\caption{Jukes-Cantor distance C code for CBMC-GC boolean circuit generation.}
\label{fig:jc_cbmc-gc}
\end{figure}
 
%\end{appendices}

\chapter{Proof of Lemma \ref{lemma:wrole_dishonest_bob} (Dishonest Bob)}
\label{app:proofBobdishonest}

As we mentioned in the main text this proof is  a combination and adaptation to our case of results from \cite{DFLSS09} and \cite{Dupuis2015}. 

To simplify the notation, in this proof we drop the subscript $0$ that refers to the RWOLE phase, e.g. we write Alice's function vector  $\mathbf{F}_0$,  simply  as $\mathbf{F}$. 


Let the values that Bob commits be fixed as  $(x_i, r_i)\, \forall i\in[m]$, where $m = (1+t)n$ and $t n$  the number of qudits $\ket{e^{x_i}_{r_i}}$ used in the \textit{Test Phase} to check whether he is honest or not. Throughout the proof we denote by $\boldsymbol{x} = (x_1, \ldots, x_m)$ and $\boldsymbol{r} = (r_1, \ldots, r_m)$ the vectors in $\mathbb{Z}_d^m$ whose components contain Bob's commitments $\forall i\in [m]$, and by $X^m=(X_1,\ldots,X_m)$ the vector of the  random variables associated to  $x_i, i\in [m]$. For each  pair $(x_i, r_i)$, the corresponding qudit $\ket{e^{x_i}_{r_i}}$ belongs in the Hilbert space $\mathcal{H}_{X_i}$, and the quantum system including all the qudits is in $\mathcal{H}_{X^m}=\bigotimes_{i\in [m]}\mathcal{H}_{X_i}$. For simplicity, we refer to the quantum systems in terms of the corresponding  random variables  $X_i$, instead of the Hilbert spaces $\mathcal{H}_{X_i}$.


Recall that the set $T\subset[m]$ contains the $tn$ indices of the test qudits, and by $\bar{T}$ we denote its complement $[m]\backslash T$.  For  $i\in T$, $\boldsymbol{x}_{|T}$ is the vector whose components are the  bases $x_i$  in which Alice will measure the test qudits, and $ \boldsymbol{r'}_{|T}$ is the vector whose components are Alice's measurement results. The corresponding quantum system is in the Hilbert space $\bigotimes_{i\in T}\mathcal{H}_{X_i}$, which for simplicity we denote as $X^m_{|T}$ in terms of the associated random variables.   
Finally, $r_H(\cdot, \cdot)=d_H(\cdot, \cdot)/n$ is the relative Hamming distance between two  vectors of size $n$, with $d_H(\cdot, \cdot)$ being their Hamming distance.


  
\begin{proof}


 Let us start by proving property 1. of Lemma \ref{lemma:wrole_dishonest_bob}.
After the first step of the protocol $\mathcal{\pi}^n_{\text{RWOLE}}$ (Figure \ref{fig:wrole}),  the generated state is $\rho_{X^m E}$, where $E$ is an auxiliary quantum system that Bob holds. Without loss of generality we assume that $\rho_{X^m E} = \ketbra{\phi_{X^m E}}$,   i.e, it is a pure state\footnote{Otherwise, we purify it and carry the purification system along with $E$.}. If Bob is honest, we have that $\ket{\phi_{X^m E}}=\ket{e^{\boldsymbol{x}}_{\boldsymbol{r}}}\otimes \ket{\psi_{E}}$, i.e.,   Bob's auxiliary quantum system $E$ is not entangled to the states that he sends to Alice.  

The \textit{Test Phase} of the protocol is used to guarantee that the real state is close to an ideal state that satisfies the properties 1. and 2. of Lemma \ref{lemma:wrole_dishonest_bob}. Let $\boldsymbol{r'}$ be the vector whose components are Alice's  outcomes when measuring the state of  $X^m$  in the committed bases $\bm{x}$, and let $\mathcal{T}$ be the random variable associated to the set of indexes $T$  of size $t n$. We can consider the state:
\begin{equation}
    \rho_{\mathcal{T}X^m E} = \rho_\mathcal{T}\otimes \ketbra{\phi_{X^m E}} = \sum_{T} P_\mathcal{T}(T) \ketbra{T}\otimes \ketbra{\phi_{X^m E}},
\label{eq:real_definition}
\end{equation}
to be the state resulting from the real execution of the protocol, and prove that it is close to some state, $\sigma_{\mathcal{T}X^m E}$, that fulfills the following property: for any choice of $T$ and for any outcome $\boldsymbol{r'}_{|T}$ when measuring the state of  ${X^m}_{|T}$ in the bases $\boldsymbol{x}_{|T}$, the relative error $r_H(\boldsymbol{r'}_{|T}, \boldsymbol{r}_{|T})$  is an upper bound on the relative error $r_H(\boldsymbol{r'}_{|\bar{T}}, \boldsymbol{r}_{|\bar{T}})$, which one would obtain by measuring the remaining subsystems ${X^m}_{|\bar{T}}$  in the bases $\boldsymbol{x}_{|\bar{T}}$. This state, $\sigma_{\mathcal{T}X^m E}$, can be  written as: 
\begin{equation}
\sigma_{\mathcal{T}X^m E} = \sum_{T} P_\mathcal{T}(T) \ketbra{T}\otimes \ketbra{\tilde{\phi}_{X^m E}^{T}},
\label{eq:ideal_definition}
\end{equation}
where $\forall T$, 
\begin{equation}
\ket{\tilde{\phi}_{X^m E}^{T}} = \sum_{\boldsymbol{r'}\in \mathcal{B}_{T}} \alpha_{\boldsymbol{r'}}^{T}\ket{e^{\boldsymbol{x}}_{\boldsymbol{r'}}}\otimes \ket{\psi^{\boldsymbol{r'}}_{E}},
\label{eq:ideal_definition_2}
\end{equation}
for  $\mathcal{B}_{T} = \{ \boldsymbol{r'}\in \mathbb{Z}_d^{m}: r_H(\boldsymbol{r'}_{|\bar{T}}, \boldsymbol{r}_{|\bar{T}})\leq r_H(\boldsymbol{r'}_{|T}, \boldsymbol{r}_{|T}) + \zeta \}$  for  $\zeta>0$ and arbitrary coefficients $\alpha^T_{\boldsymbol{r'}}$. The state $\ket{\psi^{\boldsymbol{r'}}_{E}}$ is an arbitrary state on $E$ subsystem that possibly depends on $\boldsymbol{r'}$. Then we have:

\begin{lemma}

Let the quantum states $\rho_{\mathcal{T}X^m E}$ and $\sigma_{\mathcal{T}X^m E}$   be given by \eqref{eq:real_definition} and \eqref{eq:ideal_definition}, respectively. Then, $\forall \zeta >0$ and fixed strings $\boldsymbol{x}, \boldsymbol{r}\in \mathbb{Z}_d^{m}$, we have that
\begin{equation*}
    \rho_{\mathcal{T}X^m E} \approx_\epsilon \sigma_{\mathcal{T}X^m E},
\end{equation*}
where $\epsilon(\zeta, n) = \epsilon(\zeta, n) = \exp( -\frac{2 \zeta^2t^2n^2}{(nt+1)(t+1)})$. That is, the real state $\rho_{\mathcal{T}X^m E}$ is exponentially close, with respect to $n$, to the ideal state $\sigma_{\mathcal{T}X^m E}$.
\label{lemma:dishonest_bob_2}
\end{lemma}



\begin{proof}
This proof is an adaptation of the proof of Lemma 4.3 from \cite{DFLSS09} to our case.

For any $T$, let $\ket{\tilde{\phi}_{X^m E}^T}$ be  the renormalized projection of $\ket{\phi_{X^m E}}$ into the subspace $$\text{Span}\left\{ \ket{e^{\boldsymbol{x}}_{\boldsymbol{r'}}} : \boldsymbol{r'} \in \mathcal{B}_{T} \right\}\otimes \mathcal{H}_{E},$$ and  let $\ket{\tilde{\phi}_{X^m E}^{T^\perp}}$ be the renormalized projection of $\ket{\phi_{X^m E}}$ into its orthogonal complement. 


We can, then, write
\begin{equation*}
    \ket{\phi_{X^m E}} = \epsilon_T \ket{\tilde{\phi}_{X^m E}^T} + \epsilon_T^\perp \ket{\tilde{\phi}_{X^m E}^{T^\perp}},
\end{equation*}
with  $\epsilon_T = \braket{\tilde{\phi}_{X^m E}^T}{\phi_{X^m E}}$ and $\epsilon_T^\perp = \braket{\tilde{\phi}_{X^m E}^{T^\perp}}{\phi_{X^m E}}$. By construction, this state satisfies \eqref{eq:ideal_definition_2}.

Furthermore, we can calculate the distance:
\begin{equation*}
    \delta\left( \ketbra{\phi_{X^m E}}, \ketbra{\tilde{\phi}_{X^m E}^T} \right) = \sqrt{1 - \left|\braket{\tilde{\phi}_{X^m E}^T}{\phi_{X^m E}}\right|^2} 
=\sqrt{1 - |\epsilon_T|^2} 
= |\epsilon_T^\perp|,
\end{equation*}
 where,  given $T$,  $|\epsilon_T^\perp|$ is the probability amplitude for getting  outcome $\boldsymbol{r'} \notin \mathcal{B}_{T}$ when measuring the state of $X^m$ in bases $\boldsymbol{x}$.
We continue to derive an upper bound on the distance between the real and the ideal state: 
\begin{eqnarray*}
    \delta\big( \rho_{\mathcal{T}X^m E}, \sigma_{\mathcal{T}X^m E} \big) &=& \Bigg( \sum_T P_\mathcal{T}(T) \delta\left( \ketbra{\phi_{X^m E}}, \ketbra{\tilde{\phi}_{X^m E}^T} \right) \Bigg)^2\\
    & \leq &\sum_T P_\mathcal{T}(T) |\epsilon_T^\perp|^2, 
\end{eqnarray*}
where we used Jensen's inequality and properties of the trace norm. The last term is the probability that, when choosing $T$ according to $P_{\mathcal{T}}$ and measuring the state of $X^m$ in bases $\boldsymbol{x}$ we get an outcome  $\boldsymbol{r'} \notin \mathcal{B}_{T}$. We write  
\begin{equation*}
     \sum_T P_\mathcal{T}(T)|\epsilon_T^\perp|^2 = \text{Pr}_{\mathcal{T}}[\boldsymbol{r'} \notin \mathcal{B}_{T}] = \text{Pr}_{\mathcal{T}}[r_H(\boldsymbol{r'}_{|\bar{T}}, \boldsymbol{r}_{|\bar{T}}) - r_H(\boldsymbol{r'}_{|T}, \boldsymbol{r}_{|T}) > \zeta].
\end{equation*}
 Then, we can use Lemma~\ref{lemma:trace_distance_bound} which states that the above probability is negligible in $n$ and gives us an upper bound for $\delta\big( \rho_{\mathcal{T}X^m E}, \sigma_{\mathcal{T}X^m E} \big)$. In particular, given the set $[m]$ with $(1+t)n$ elements, we apply the aforementioned corollary for a random subset $\mathcal{T}$ of size $tn$ and its complement $\bar{\mathcal{T}}$ of size $n$. Denoting by $\mu_{\mathcal{T}}$ and $\mu_{\bar{\mathcal{T}}}$, respectively,  the averages of these subsets, we obtain
\begin{equation*}
\text{Pr}[\mu_{\bar{\mathcal{T}}} - \mu_{\mathcal{T}} \geq \zeta] \leq \exp( -\frac{2 \zeta^2t^2n^2}{(nt+1)(t+1)}).
    \end{equation*}

Hence, we have:
\begin{equation}
    \delta\big( \rho_{\mathcal{T}X^m E}, \sigma_{\mathcal{T}X^m E} \big) \leq  \sum_T P_\mathcal{T}(T)|\epsilon_T^\perp|^2 \leq \exp( -\frac{2 \zeta^2t^2n^2}{(nt+1)(t+1)})=: \epsilon,
\end{equation}
concluding the proof of Lemma \ref{lemma:dishonest_bob_2}, i.e. that the real state \eqref{eq:real_definition} generated by the protocol until the \textit{Computation Phase}  is  $\epsilon-$close to the ideal state \eqref{eq:ideal_definition}.
\end{proof}

It is now straightforward to complete the proof of property 1. of Lemma \ref{lemma:wrole_dishonest_bob}. 

To obtain the states $\sigma_{\mathbf{F} B'}$ and $\rho_{\mathbf{F} B'}$ from the states $\sigma_{\mathcal{T}X^m E}$ and $\rho_{\mathcal{T}X^m E}$, respectively, we need to apply the operator $V^{\bm{b}}_{\bm{a}}\Tr_{X^m_{|T}\,E_{|T}}[\, \cdot\,] V^{\bm{b} \dagger}_{\bm{a}}$. This operator  is a CPTP map, being the composition of the two CPTP maps, $V^{\bm{b}}_{\bm{a}}$ and $\Tr$. Since the trace distance between two density matrices does not increase under CPTP maps (see Lemma 7 in \cite{U17}), the final states indeed satisfy  property 1. of Lemma \ref{lemma:wrole_dishonest_bob}, namely 
\begin{equation*}
    \sigma_{\mathbf{F} B'} \approx_{\epsilon} \rho_{\mathbf{F} B'}.
\end{equation*}



 For the rest of this proof dishonest Bob's system $B'$ is identified with $\mathbf{Y}\, E$, where $\mathbf{Y}$ corresponds to the classical information leaking to him through the output of the WROLE and $E$ is, in general, a quantum auxiliary system that he might also hold. Consequently, from now on we write $\sigma_{\mathbf{F} B'}$ as $\sigma_{\mathbf{F} \mathbf{Y}E}$. Now, we have to prove property 2. of Lemma~\ref{lemma:wrole_dishonest_bob},  i.e., obtain the corresponding lower bound on min-entropy with respect to $\sigma_{\mathbf{F} \mathbf{Y}E}$. 
  We start with  the following Lemma~\ref{XnBound}:


 \begin{lemma}[Corollary 4.4 in \cite{DFLSS09}]
\label{XnBound}
Let $err := r_H(\boldsymbol{r'}_{|T}, \boldsymbol{r}_{|T}) \leq 1-\frac{1}{d}$ be the error measured by Alice  while measuring the state of $X^m_{|T}$ according to her choice of $T$, and let $\sigma_{\mathbf{X} E} := \ketbra{\psi}_{\mathbf{X} E}$ be the state to which the ideal state $\sigma_{\mathcal{T}X^m E}$ collapses after this measurement. Following \eqref{eq:ideal_definition} and \eqref{eq:ideal_definition_2}, we  write  $\ket{\psi}_{\mathbf{X} E} = \sum_{\boldsymbol{z}\in \mathcal{B}} \alpha_{\boldsymbol{z}}\ket{e^{\boldsymbol{x}}_{\boldsymbol{z}}} \ket{\psi^{\boldsymbol{z}}_{E}}$  for some $\ket{\psi^{\boldsymbol{z}}_{E}}$ and  $\mathcal{B} = \{ \boldsymbol{z}\in \mathbb{Z}_d^{n}: r_H(\boldsymbol{z}, \boldsymbol{r}_{|\bar{T}})\leq err + \zeta \}$ with $\zeta>0$. Then, we have: 
\begin{equation}
H_{\min}(\mathbf{X}|E)_{\sigma_{\mathbf{X} E}} \geq -h_d(err + \zeta) n\log d,
\label{eq:min_entropy_1_stage}
\end{equation}
where $h_d(x)$ is given in Definition \ref{def:q-ary}.

\end{lemma}
\begin{proof}

 We start by defining the state $\tilde{\sigma}_{\mathbf{X} E}: = \sum_{\bm{z}\in \mathcal{B}} |\alpha_{\bm{z}}|^2 \ketbra{e^{\boldsymbol{x}}_{\boldsymbol{z}}} \otimes \ketbra{\psi^{\boldsymbol{z}}_{E}}$. 
Then, by applying Lemma~\ref{lemma:renner_lower_bound}, we obtain
\begin{equation*}
    H_{\text{min}}(\mathbf{X} |E)_{\sigma_{\mathbf{X} E}} \geq H_{\text{min}}(\mathbf{X} | E)_{\tilde{\sigma}_{\mathbf{X} E}} - \log |\mathcal{B}|.
\end{equation*}


Since $\tilde{\sigma}_{\mathbf{X} E}$  is a classical-quantum state, its min-entropy cannot be negative, i.e. $$H_{\text{min}}(\mathbf{X} | E)_{\tilde{\sigma}_{\mathbf{X} E}} \geq 0,$$ thus $H_{\text{min}}(\mathbf{X} |E)_{\sigma_{\mathbf{X} E}} \geq -\log |\mathcal{B}|$. 

Finally, to get the lower bound shown in \eqref{eq:min_entropy_1_stage}, we apply Lemma~\ref{lemma:hammingBall} to our case, namely for the Hamming ball around $\boldsymbol{r}_{|\bar{T}}$ with radius $n(err + \zeta)$.

\end{proof}

To complete the proof of property 2. of Lemma \ref{lemma:wrole_dishonest_bob} and find a lower bound on $H_\text{min}(\mathbf{F}|\mathbf{Y}\, E)\sigma_{\mathbf{F} \mathbf{Y}E}$  we need to relate it with $H_{\text{min}}(\mathbf{X} |E)_{\sigma_{\mathbf{X} E}}$, for which we just derived a lower bound. In what follows, we 
  adapt the notation from \cite{Dupuis2015}:

\begin{itemize}
    \item $\Phi_{\mathbf{X} \bar{\mathbf{X}}} = \ketbra{\Phi}_{\mathbf{X} \bar{\mathbf{X}}}$, where $\ket{\Phi}_{\mathbf{X} \bar{\mathbf{X}}} = \sum_{\bm{s}} \ket{e^{\bm{x}}_{\bm{s}}}_{\mathbf{X} } \otimes \ket{e^{\bm{x}}_{\bm{s}}}_{\bar{\mathbf{X}}}$ for all basis choices   $\bm{x}  \in\mathbb{Z}^n_d$, and
    \item $\Phi_{(\bm{a},\bm{b})} = \ketbra{\Phi_{(\bm{a},\bm{b})}} = \sum_{\bm{s s'}} V^{\bm{b}}_{\bm{a}} \ketbra{e^{\bm{x}}_{\bm{s}}}{e^{\bm{x}}_{\bm{s'}}} V^{\bm{b}\dagger}_{\bm{a}} \otimes \ketbra{e^{\bm{x}}_{\bm{s}}}{e^{\bm{x}}_{\bm{s'}}}$, with $\ket{\Phi_{(\bm{a},\bm{b})}} = (V^{\bm{b}}_{\bm{a}} \otimes \mathds{1}) \ket{\Phi}_{\mathbf{X} \bar{\mathbf{X}}}$,  for $(\bm{a},\bm{b})\in\mathbb{Z}^2_d$
\end{itemize}
and we use the following properties of the operators $V_a^b$, which can be derived from \eqref{eq:main_relation}:

\begin{enumerate}%[(a)]
    \item $V^b_a \ketbra{e^x_i} V^{b \dagger}_a = \ketbra{e^x_{ax - b + i}}$, \label{prop:a}
    \item $V^b_a = \sum_j c_{a,b,j,x} \ketbra{e^x_{ax - b + j}}{e^x_j}$,\label{prop:b}
    \item $V^{b\dagger}_a = \sum_j c_{a,b,j,x}^* \ketbra{e^x_j}{e^x_{ax - b + j}}$.\label{prop:c}
\end{enumerate}
The following lemma,  which is an adaptation of Theorem 12 in \cite{Dupuis2015} to our case, provides a lower bound for $H_\text{min}(\mathbf{F}|\mathbf{Y}\, E)_{\sigma_{\mathbf{F}\mathbf{Y} E}}$ in terms of $H_{\text{min}}(\mathbf{X} |E)_{\sigma_{\mathbf{X}E}}$:
\begin{lemma} 
\label{lem:GnBound}
Let $\mathbf{X}$ denote our $n$-qudit system and $\sigma_{\mathbf{X} E}$ be the ideal quantum state to which the system collapsed after Alice's test measurements, as introduced before. Then, we have:
\begin{equation*}
    \text{H}_\text{min}(\mathbf{F}|\mathbf{Y}\, E)_{\sigma_{\mathbf{F}\mathbf{Y} E}} \geq \frac{1}{2} \big(n\log d + \text{H}_\text{min}(\mathbf{X}| E)_{\sigma_{\mathbf{X} E}}\big),
\end{equation*}
where 
\begin{equation}
    \sigma_{\mathbf{F}\mathbf{Y} E} = \frac{1}{d^{2n}} \sum_{(\bm{a}, \bm{b}) \in \mathbb{Z}^{2n}_d} \ketbra{e^{\bm{x}}_{\bm{a}},e^{\bm{x}}_{\bm{b}}}\otimes V^{\bm{b}}_{\bm{a}} \, \sigma_{\mathbf{X} E}\,  V^{\bm{b}\dagger}_{\bm{a}},
    \label{eq:VappliedX}
\end{equation}
is the state obtained when $V^{\bm{b}}_{\bm{a}}$ is applied to the system $\mathbf{X}$ according to $\mathbf{F}$. 
\end{lemma}

\begin{proof}
This proof is an adaptation of the proof of Theorem 12 in \cite{Dupuis2015} to our case.
Let us fix $x\in\mathbb{Z}_d$ and write $\ket{i} = \ket{e^x_i}$  for short. 
 $V^b_a$ is a CPTP map, and it is known that the min-entropy does not decrease whenever a CPTP map is applied. However, this is not enough  to prove the security of the protocol and determine a lower bound. We need a more refined expression relating  $H_{\min}(\mathcal{M}(\mathbf{X})|E)$ and  $H_{\min}(\mathbf{X}|E)$ for some CPTP map $\mathcal{M}$. This is given by Lemma~\ref{thm:entaglementSamplingResult}, which can be applied for $$\mathcal{M}_{\mathbf{X}\rightarrow \mathbf{F}\mathbf{Y}} = \mathcal{N}^{\otimes n},$$ where $\mathcal{N}(\sigma_X) = \frac{1}{d^2}\sum_{(a,b)\in\mathbb{Z}^2_d}\mathcal{N}_{a,b} \sigma_X \mathcal{N}_{a,b}^\dagger = \frac{1}{d^2} \sum_{(a,b)\in\mathbb{Z}^2_d} \bigg( \ket{a,b} \otimes V^b_a \bigg) \sigma_X \bigg( \bra{a,b} \otimes V^{b\dagger}_a \bigg) $. The operator $\mathcal{N}$ applies the operator $V^b_a$ to the single system $X$ and saves its choice $(a,b)$ to a new record in the  $F$ space.  

To use Lemma~\ref{thm:entaglementSamplingResult}, we have to prove that $((\mathcal{M}^\dagger \circ \mathcal{M} )_{\mathbf{X}} \otimes \text{id}_{\bar{\mathbf{X}}}) (\Phi_{\mathbf{X} \bar{\mathbf{X}}})$ can be written as a linear combination of $\Phi_{(\bm{a},\bm{b})}$, i.e.
$$((\mathcal{M}^\dagger \circ \mathcal{M} )_{\mathbf{X}} \otimes \text{id}_{\bar{\mathbf{X}}}) (\Phi_{\mathbf{X} \bar{\mathbf{X}}}) = \sum_{(\bm{a},\bm{b})\in \mathbb{Z}^{2n}} \lambda_{(\bm{a},\bm{b})} \Phi_{(\bm{a},\bm{b})}.$$

First, note that 
\begin{align}
    \mathcal{N}(\ketbra{i}{j}) &= \frac{1}{d^2} \sum_{(a,b)\in\mathbb{Z}^2_d} \bigg( \ket{a,b} \otimes V^b_a \bigg) \ketbra{i}{j} \bigg( \bra{a,b} \otimes V^{b\dagger}_a \bigg) \nonumber\\
    \alignedoverset{\ref{prop:a}}{=} \frac{1}{d^2} \sum_{(a,b)\in\mathbb{Z}^2_d} \ketbra{a,b} \otimes \ketbra{ax-b+i}{ax-b+j}, \label{eqn:Nresult}
\end{align}

where we used the property \ref{prop:a}.  We proceed to compute $\mathcal{N}^\dagger \circ \mathcal{N} \ketbra{i}{j}$:
\begin{align}
\mathcal{N}^\dagger \circ \mathcal{N} \ketbra{i}{j} &= \frac{1}{d^2} \sum_{(a',b')\in\mathbb{Z}^2_d} \mathcal{N}^\dagger_{a' b'} (\mathcal{N}\ketbra{i}{j})) \mathcal{N}_{a' b'} \nonumber\\
\alignedoverset{\text{eq.(\ref{eqn:Nresult})}}{=} \frac{1}{d^4} \sum_{(a',b'),(a,b)\in\mathbb{Z}^2_d} \Big(\bra{a',b'} \otimes V^{b'\dagger}_{a'} \Big) \ketbra{a,b} \nonumber \\
&\hspace{4.5cm}\otimes \ketbra{ax-b+i}{ax-b+j} \Big( \ket{a',b'} \otimes V^{b'}_{a'}  \Big) \nonumber\\
\alignedoverset{\ref{prop:b}, \ref{prop:c}}{=} \frac{1}{d^4}\sum_{(a,b)\in\mathbb{Z}^2_d} V^{b\dagger}_{a} \ketbra{ax-b+i}{ax-b+j} V^b_a \nonumber\\
&= \frac{1}{d^4} \sum_{(a,b)\in\mathbb{Z}^2_d} \ketbra{i}{j} \nonumber\\
&= \frac{1}{d^2} \ketbra{i}{j}, \label{eqn:NdaggerNresult} 
\end{align}
and
\begin{align}
    ((\mathcal{N}^\dagger \circ \mathcal{N})_{X} \otimes \text{id}_{\bar{X}}) (\Phi_{X \bar{X}}) &= \sum_{i,j} \mathcal{N}^\dagger \circ \mathcal{N} (\ketbra{i}{j}) \otimes \ketbra{i}{j} \nonumber\\
\alignedoverset{\text{eq.}(\ref{eqn:NdaggerNresult}) }{=} \sum_{i,j} \Big(  \frac{1}{d^2} \ketbra{i}{j} \Big) \otimes \ketbra{i}{j} \nonumber\\
&= \frac{1}{d^2} \sum_{i,j} \ketbra{i}{j} \otimes \ketbra{i}{j} \nonumber\\
&= \frac{1}{d^2} \Phi_{X \bar{X}}.\label{eqn:NdaggerNidresult}
\end{align}
Therefore, we have that 
\begin{equation}
    \big((\mathcal{N}^\dagger \circ \mathcal{N})_{X} \otimes \text{id}_{\bar{X}}\big) (\Phi_{X \bar{X}}) = \frac{1}{d^2} \Phi_{(0,0)},
\end{equation}
from which we easily see that 
\begin{equation}
    \big((\mathcal{M}^\dagger \circ \mathcal{M} )_{\mathbf{X}} \otimes \text{id}_{\bar{\mathbf{X}}}\big) (\Phi_{\mathbf{X} \bar{\mathbf{X}}}) = \frac{1}{d^{2n}} \Phi_{(\bm{0},\bm{0})}.
\end{equation}
Consequently, 
\begin{equation*}
    \lambda_{(\bm{a}, \bm{b})}= \left\{
\begin{array}{ll}
      \frac{1}{d^{2n}} & \text{if } (\bm{a}, \bm{b}) = (\bm{0},\bm{0})\\
      0 & \text{otherwise.} \\
\end{array} 
\right. 
\end{equation*}

Now, we want to choose the partition that gives us the best lower bound on the collision entropy, i.e. decreases the r.h.s of the following relation:
\begin{equation*}
2^{-\text{H}_2(\mathbf{F}\mathbf{Y} | E)_{\sigma_{\mathbf{F}\mathbf{Y} E}}} \leq \sum_{(\bm{a},\bm{b}) \in \mathfrak{S}_+} \lambda_{(\bm{a},\bm{b})} 2^{-\text{H}_2(\mathbf{X}| E)_{\sigma_{\mathbf{X} E}}} + (\max_{(\bm{a},\bm{b}) \in \mathfrak{S}_-} \lambda_{(\bm{a},\bm{b})}) d^n.
\end{equation*}

Note that  we dropped the conditioning on the state $\sigma_{\mathbf{X}E}$ at $\text{H}_2(\mathbf{F}\mathbf{Y} | E)_{\sigma_{\mathbf{F}\mathbf{Y} E}}$. This is because the map $\mathcal{M}$ is trace-preserving (for a detailed explanation see \cite{Dupuis2015} below Theorem 1). For our case there are just two types of partitions: the case where $0 \in \mathfrak{S}_+$ and the case where $0 \in \mathfrak{S}_-$. If $0 \in \mathfrak{S}_+$:
\begin{equation*}
\text{r.h.s} = \sum_{(\bm{a}, \bm{b}) \in \mathfrak{S}_+} \lambda_{s} 2^{-\text{H}_2(\mathbf{X}| E)_{\sigma_{\mathbf{X} E}}} = \frac{1}{d^{2n}} 2^{-\text{H}_2(\mathbf{X}| E)_{\sigma_{\mathbf{X} E}}} \leq \frac{1}{d^n}.
\end{equation*}
The last inequality holds because $-n\log d \leq \text{H}_2(\mathbf{X}| E)_{\sigma_{\mathbf{X} E}} \leq n\log d$. In fact,
\begin{equation*}
 \frac{1}{d^{2n}} 2^{-\text{H}_2(\mathbf{X}| E)_{\sigma_{\mathbf{X} E}}} \leq \frac{1}{d^n}
\iff 2^{-\text{H}_2(\mathbf{X}| E)_{\sigma_{\mathbf{X} E}}} \leq d^n 
\iff \text{H}_2(\mathbf{X}| E)_{\sigma_{\mathbf{X} E}}\geq -n\log d.
\end{equation*}

If $0 \in \mathfrak{S}_-$, $\text{r.h.s} = \frac{1}{d^n}$ which, as we have just seen by the previous inequality, does not provide a better lower bound on the collision entropy whenever $\text{H}_2(\mathbf{X}| E)_{\sigma_{\mathbf{X} E}} \neq -n\log d$.

So, choosing any partition such that $0 \in \mathfrak{S}_+$, we get
\begin{equation*}
2^{-\text{H}_2(\mathbf{F}\mathbf{Y} | E)_{\sigma_{\mathbf{F}\mathbf{Y} E}}} \leq \sum_{(\bm{a}, \bm{b}) \in \mathfrak{S}_+} \lambda_{(\bm{a}, \bm{b})} 2^{-\text{H}_2(\mathbf{X}| E)_{\sigma_{\mathbf{X} E}}} + (\max_{(\bm{a}, \bm{b}) \in \mathfrak{S}_-} \lambda_{(\bm{a}, \bm{b})}) d^n =\frac{1}{d^{2n}} 2^{-\text{H}_2(\mathbf{X} | E)_{\sigma_{\mathbf{X} E}}},
\end{equation*}
from which we conclude that
\begin{equation}
\text{H}_2(\mathbf{F}\mathbf{Y} | E)_{\sigma_{\mathbf{F}\mathbf{Y} E}} \geq 2n\log d + \text{H}_2(\mathbf{X}|E)_{\sigma_{\mathbf{X} E}}.  \label{eqn:firstineq} 
\end{equation}

In order to relate $\text{H}_2(\mathbf{F}\mathbf{Y} | E)_{\sigma_{\mathbf{F}\mathbf{Y} E}}$ with $\text{H}_2(\mathbf{F} | \mathbf{Y} E)_{\sigma_{\mathbf{F}\mathbf{Y} E}}$, we use the chain rule given by LEmma~\ref{lemma:chain_rule}:
\begin{equation}
\text{H}_2(\mathbf{F} | \mathbf{Y} E)_{\sigma_{\mathbf{F}\mathbf{Y} E}} \geq \text{H}_2(\mathbf{F}\mathbf{Y} | E)_{\sigma_{\mathbf{F}\mathbf{Y} E}} - \log \text{rank}(\sigma_{\mathbf{Y}}) \geq \text{H}_2(\mathbf{F}\mathbf{Y} | E)_{\sigma_{\mathbf{F}\mathbf{Y} E}} - n \log d,
 \label{eqn:chainRule} 
\end{equation}
since $\log \text{rank}(\sigma_{\mathbf{Y}}) \leq n \log d$. Combining (\ref{eqn:firstineq}) and (\ref{eqn:chainRule}), we get the desired result:

\begin{equation*}
    \text{H}_2(\mathbf{F}|\mathbf{Y}\, E)_{\sigma_{\mathbf{F}\mathbf{Y} E}} \geq n\log d + \text{H}_2(\mathbf{X}| E)_{\sigma_{\mathbf{X} E}}.
\end{equation*}
To express the above relation in terms of the min-entropy instead of the collision entropy, we start by noticing that 
 the state given in \eqref{eq:VappliedX} can be written as 
\begin{align}
  \sigma_{\mathbf{F}\mathbf{Y} E} &= \frac{1}{d^{2n}} \sum_{(\bm{a}, \bm{b}) \in \mathbb{Z}^{2n}_d} \ketbra{\bm{a},\bm{b}}\otimes V^{\bm{b}}_{\bm{a}} \, \sigma_{\mathbf{X} E}\,  V^{\bm{b}\dagger}_{\bm{a}} = \frac{1}{d^{2n}} \sum_{(\bm{a}, \bm{b}) \in \mathbb{Z}^{2n}_d} \ketbra{\bm{a},\bm{b}}\otimes \sigma_{\mathbf{X} E}^{\bm{a},\bm{b}},
\end{align}
which is a classical-quantum state. Therefore, we can use Lemma 18 from \cite{Dupuis2015} to obtain
\begin{equation*}
     \text{H}_{\text{min}}(\mathbf{F}|\mathbf{Y}\, E)_{\sigma_{\mathbf{F}\mathbf{Y} E}}  \geq \frac{1}{2}\left(n\log d + \text{H}_2(\mathbf{X}| E)_{\sigma_{\mathbf{X} E}}\right).
\end{equation*}
Furthermore,  $\sigma_{\mathbf{X}E}$  is a general quantum state, and from Lemma 17 in \cite{Dupuis2015} we have 
\begin{equation*}
    \text{H}_{\text{min}}(\mathbf{F}|\mathbf{Y}\, E)_{\sigma_{\mathbf{F}\mathbf{Y} E}}  \geq \frac{1}{2}\left(n\log d + \text{H}_{\text{min}}(\mathbf{X}| E)_{\sigma_{\mathbf{X} E}} \right).
\end{equation*}
\end{proof}



To complete the proof of property 2. of Lemma \ref{lemma:wrole_dishonest_bob}, we  combine Lemma~\ref{XnBound} and Lemma~\ref{lem:GnBound}  and obtain:
\begin{equation*}
H_{\min}(\mathbf{F} | \mathbf{Y} E)_{\sigma_{\mathbf{F}\mathbf{Y} E}} \geq \frac{1}{2}(n\log d - h_d(\zeta)n\log d) = \frac{n\log d}{2}(1 - h_d(\zeta)),
\end{equation*}
for $err = 0$.

\end{proof}

%
%
%
%\bibliography{bibforthesis}
%\bibliographystyle{unsrt}
%
%\end{document}