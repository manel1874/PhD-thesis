%\documentclass[11pt]{report}



%\begin{document}

\chapter{Introduction}
%\addcontentsline{toc}{chapter}{Introduction}

\section{Motivation}

Since the invention of writing, the need for secret communication resulted in the development of cryptography -- the art of ``hidden communication''.  It started by using simple symbols as code words and evolved to the stage where the security is based on various mathematical hardness assumptions: the widely used RSA-based cryptographic system\cite{riv:sha:adl:78} relies on the conjecture that factoring large numbers is not feasible using standard computers, while the alternative lattice-based public-key cryptographic system\cite{gol:gol:hal:97} is based on the assumed difficulty of the so-called ``shortest and closest vector problems'' (also related to the well known $P\neq NP$ conjecture\cite{fort:13}). 
%
With the advent of quantum computation, and in particular after the discovery of the celebrated Shor's algorithm for the efficient factoring of prime numbers\cite{sho:97}, the security of numerous cryptographic systems currently in use became jeopardised, and as a consequence the need for new cryptographic systems resilient to quantum adversaries arose. The above mentioned lattice-based cryptographic system is resilient to quantum adversaries that execute Shor's algorithm, as its security is not based on the factoring problem; nevertheless, it does rely on another mathematical assumption, thus it is just computationally  secure. 
In this context, the idea of quantum cryptography was born. The security of the communication, rather than relying on mathematical/ computational hardness assumptions, is now based on the laws of quantum mechanics, i.e, quantum cryptographic protocols are unconditionally secure (information-theoretic security). The only classical encryption scheme which is known to be unconditionally secure is the one-time pad, however the key management is a very hard task. The major advantage of quantum cryptography is that it is not only unconditionally secure, but also the key management is easy. Quantum cryptography was first considered by Wiesner in the late sixties and early seventies, who introduced the notions of quantum multiplexing and money (this work, though, was only published a decade later in 1983~\cite{wie:83}), and further developed in 1984 by Bennet and Brassard~\cite{ben:bra:84} in their famous Quantum Key Distribution (QKD) BB84 protocol. Subsequently, Shor and Preskill~\cite{sho:pre:00} and independently Mayers~\cite{may:01} showed that the BB84 protocol is unconditionally secure. Since then, quantum cryptography has been among the hottest subjects of research and the intense investigation has yielded so far impressive results. Several QKD experiments over long distances have been reported~\cite{urs:etal:07,sch:07,tak:sas:tam:koa:15,pug:kai:bou:jin:sul:agn:ani:mak:cho:hig:jen:17,idquantique}, and QKD is already commercial.\footnote{Currently there are three companies offering commercial QKD systems: ID Quantique (Geneva), MagiQ Technologies, Inc. (New York) and QuintessenceLabs (Australia).} Furthermore, the recent successful launch of a satellite~\cite{yin:etal:17} paved the way for intercontinental QKD~\cite{vie:bei:qkd:sat:18}. 

In the first part of this thesis (Chapters 2 and 3), we present our work on quantum cryptography. In particular, we  propose a new secure quantum public-key cryptosystem, as an alternative to the currently used classical public-key cryptosystems, whose security can be seriously compromised by adversaries with potential access to a quantum computer. Moreover, we design three novel QKD protocols and study their security properties. This way, we stay in tune with the current advances in QKD, which is widely considered as the most secure and practical instance of quantum cryptography, so far. 

The cryptographic protocols that we propose (public-key cryptosystem and key distribution) have a common feature. They are all based on quantum walks (QWs), the quantum counterpart of classical random walks. A classical random walk describes the behaviour of a ``walker'' over a path who, at each step, can choose to follow one of the possible directions with a certain {\em a priori} fixed probability. It was shown to be very useful in computer science (sampling massive online graphs, image segmentation, estimating the size of the World Wide Web, wireless networking, etc.), physics (modelling the Brownian motion, studying polymers, etc.), and many other fields of research (financial economics, medicine and biology, psychology, etc.).
One may study several properties of classical random walks, such as the probability of returning to the original position after a certain number of steps, the characteristics of the probability distribution of the positions at each step, etc...

QWs were introduced in 1993~\cite{aha:dav:zag:93}, as the quantum analogue of classical random walks and since then, they have been playing a prominent role in quantum computation. Unlike the classical case, in which the state of the walker is described by a probability distribution over the allowed positions, in the quantum scenario the state of the walker is given by a superposition of positions. One can study different types of QWs, determined by their time evolution (discrete- vs continuous-time) and the topology of the underlying positions space (walks on the line, lattice, circle, graphs, etc.).
They are a very successful tool in algorithmic theory, since they provide polynomial and exponential speedups over classical computations for several problems~\cite{far:gut:98,chi:etal:03,kem:03,amb:03}. They have also been proven to be very useful in search problems~\cite{san:08,cha:nov:amb:oma:16,nov:cha:moh:nev:oma:16,por:13}. Moreover, they are an important computational primitive, since they permit universal quantum computation \cite{chi:09,lov:coo:eve:tre:ken:10,chi:gos:web:13}.

Recently, the use of QWs for cryptographic purposes has been also suggested. For instance, in \cite{roh:fit:gil:15}, Rohde {\em et al.}, proposed a limited form of quantum homomorphic encryption using multi-particle QWs. In their protocol, a server could manipulate data sent by a client in such a way that the server has limited information on the client's data, while the client has limited information on the server's computation. Also, Yan {\it et al} presented a new method to generate keys for image encryption using QWs~\cite{yan:pan:sun:xu:15}.
To the best of our knowledge, our work is the first attempt to use QWs for the encryption of messages and key distribution and perhaps its most important contribution is that it introduces this exciting possibility, which may spur new research in both cryptography and the study of QWs. By themselves, QWs exhibit many fascinating properties which, as we show, translate to interesting properties of quantum cryptographic protocols.

An issue with significant impact on both classical and quantum cryptography is the absence of long-term stable quantum memories. The design of quantum memories, in which a large amount of quantum states should be stored for a long time and processed, is a hard task, since decoherence effects, due to thermal noise and imperfections of the system, inevitably occur.\footnote{During the completion of this thesis, we became aware of two very impressive and promising recent results. In~\cite{warshaw:mem:17}, the experimental demonstration of the simultaneous storing of 665 quantum states for more than $50\mu s$ was reported and in~\cite{vienna:mem:18} the storing of a qubit for 8 hours was achieved.} This is a huge challenge towards the construction of operational and scalable quantum computers.
Therefore, the classical cryptosystems that are vulnerable to quantum adversaries remain secure as long as a large-scale quantum computer does not exist. On the other hand, the security of quantum cryptosystems depends also on the existence or absence of long-term stable quantum memories, as we will now explain. In general, the security proofs of quantum cryptographic protocols assume all-powerful adversaries who are able to reliably store quantum states for as long as they wish~\cite{ber:chr:col:ren:ren:10}. Their security, thus, will not be compromised by a future technological development of stable quantum memories. However, for practical mid-term applications of quantum cryptography, it is sufficient to consider adversaries with access to more realistic noisy and/ or bounded quantum memories. Along these lines, there have been proposed several cryptographic protocols, whose security is studied with respect to the memory resources that an adversary could possess~\cite{dam:feh:ren:sal:sch:07,weh:sch:ter:08,sch:ter:weh:09,ng:jos:che:kur:weh:12,koe:weh:wul:12,bou:feh:gon:sch:13,lou:alm:and:pin:mat:pau:14}.
   
In search of stable quantum memories that would, in turn, permit the realisation of large-scale quantum computers, researchers from different areas have focused their attention in trying to find physical systems able to preserve the coherence of quantum states for timescales much longer than the timescale of logical operations. Topologically ordered many-body systems are among the best candidates to fulfil this requirement.
These systems have attracted a lot of interest during the last few decades, as they possess some unique ``exotic'' properties~\cite{has:kan:10, qi:zha:11, ber:hug:13}, which have potentially many applications in various emerging fields such as spintronics, photonics and -- as already mentioned -- quantum computing. There are two main reasons for which topological systems are very promising for designing stable quantum memories and for performing fault-tolerant quantum computation. First, because their degenerate ground states cannot be distinguished by local observables. This local indistinguishability of quantum information implies that the effects due to local noise can be reversible.  Furthermore, it has been shown that the topological order is robust against weak local perturbations on the Hamiltonian at zero temperature~\cite{nay:sim:ste:fre:das:08,kit:03,bra:has:mic:10}. These two properties, though, do not solve the problem of decoherence due to thermal noise, which is always present, as maintaining a many-body system at zero temperature for a long time is practically unfeasible.
For this purpose, a lot of studies have been concentrated on the behaviour of topological order at finite temperatures. As a result, there have been stated several no-go theorems, that rule out some specific types of many-body lattice Hamiltonians with topological features, which definitely fail to maintain coherence at finite temperatures~\cite{bro:los:pac:sel:woo:16}. In parallel, several new physical models have been proposed, whose properties permit the protection of quantum information in thermal environments (for a complete and detailed review, see~\cite{bro:los:pac:sel:woo:16}). Most of these models can be decomposed into simpler ones which exhibit topological behaviour and they are based on the seminal toric code~\cite{kit:03}, a landmark in quantum error correction and the topological quantum computing paradigm.

In the second part of this thesis (Chapters 4, 5 and 6), we present our work on the behaviour of topological order at finite temperatures. In particular, we probe the robustness of the topological features with respect to temperature for paradigmatic models of topological systems, by means of the study of the corresponding phase transitions in and out of equilibrium. This way, we aim to contribute to the current effort of fully understanding the properties of topological systems, which are significant for the future design of stable quantum memories and subsequently large-scale quantum computers, with important implications in classical and quantum cryptography.
 
In the following sections of this chapter, we present general, brief introductions to all the aforementioned topics, thus putting into a more concrete context the work developed in this thesis.  
\section{Public-key cryptography}
 
Public-key cryptography (also called asymmetric cryptography) refers to cryptographic systems that use pairs of public and private (or secret) keys. The public keys, as their name suggests, are known and they are shared among all the users involved in a certain communication protocol, while the corresponding private keys are only known to the owner. 
Public-key cryptosystems can be used for message encryption or for authentication. To illustrate how they work, let us consider the following examples:
\begin{itemize}
\item \textit{Message encryption}\\
Suppose that two parties Alice ($A$) and  Bob ($B$) want to exchange a message. $A$ uses her private key to generate her public key, which she sends over a public channel to $B$. $B$ uses this public key to encrypt his message and sends it to $A$. Upon receiving it, $A$ uses her private key to decrypt $B$'s message. It is straightforward to understand that the security of the communication protocol depends on the secrecy of the private key, since anyone can encrypt using the public key, but only one can decrypt using the private key.  

\item \textit{Authentication}\\
Suppose that $B$ received an encrypted message by $A$ and wants to verify that it was actually her who sent it. $A$ has signed the message with her private key and $B$ can use the associated public key to verify that it was $A$, i.e., the owner of the private key, who sent the message.
\end{itemize}
The security of most public-key cryptosystems relies on the so-called trapdoor one-way functions. 

\begin{definition} (Honest function)\\
	A function $f:\mathbb{N}\rightarrow \mathbb{N}$ is called honest if $|f(x)|$ and $|x|$ are polynomially related, i. e., there exists a $k$ such that for all $x\in\mathbb{N}$:
	$$|f(x)|\leq |x|^k +k$$ and $$|x|\leq |f(x)|^k +k.$$
\end{definition}

\begin{definition} (One-way function)\\
An honest function $f:\{0,1\}^*\rightarrow \{0,1\}^*$ is called one-way if it is injective and the following two conditions hold:\\
1)(Easy to compute): There exists a polynomial time algorithm $\mathcal{A}$, such that on input $x$, $\mathcal{A}$ outputs $f(x)$ (i.e. $\mathcal{A}(x)=f(x)$).\\
2)(Hard to invert): For every probabilistic polynomial time algorithm $\mathcal{A}'$, every polynomial $\poly$ and all sufficiently large $n=length(x):$
$$Pr[f(\mathcal{A}'(f(x)))=f(x)]<1/\poly(n).$$
\end{definition}
If we include the extra requirement that the function becomes easy to invert given some extra information called the trapdoor information, then the function is called a {\em trapdoor one-way function}. 

Public-key cryptosystems take advantage of these properties of one-way functions to ensure security of the communication. In particular, they provide the legitimate users $A$ and $B$ with a mathematical problem which is easy to solve, while any eavesdropper $E$, attempting to intercept, needs to solve a computationally very hard problem. 
There exist several computationally hard mathematical problems, which are good candidates for one-way functions, such as the integer factorisation problem and the discrete logarithm problem, upon which most of the practical cryptosystems that we use nowadays are based. Nevertheless, a rigorous proof for the existence of one-way functions is missing; such a proof would also imply the solution of the famous open problem $P\neq NP$~\cite{fort:13}. Practically, this means that the security of all current cryptosystems based on mathematical hardness assumptions can be threatened by adversaries who might possess advanced algorithms and hardware. Such an example is the celebrated Shor's quantum algorithm for the  efficient factoring of prime numbers~\cite{sho:97}. It was proposed in 1997 by Peter Shor and it solves in polynomial time the integer factoring and the discrete logarithm problem. Hence, when quantum computers will be available, an adversary that possesses one will be able to crack almost all practical public-key cryptosystems (such as RSA, Diffie-Hellman, ElGamal etc). Consequently, the need for public-key cryptosystems secure against attacks by adversaries with quantum computers arose. In a first attempt to address this problem, the authors in~\cite{oka:tan:uch:00} extend the model of public-key cryptosystems to quantum public-key cryptosystems. They  define quantum trapdoor one-way functions, as the counterpart of trapdoor one-way functions in the case of quantum Turing machines and assuming their existence they define quantum public-key cryptographic systems, in analogy to the classical case. Later, in~\cite{kaw:etal:05} a concrete quantum public-key cryptographic system was presented, whose security is based on the indistinguishability of quantum states and more recently, Nikolopoulos~\cite{nik:08,sey:nik:alb:12}  proposed a secure public-key encryption scheme based on single-qubit rotations. 



\section{Quantum key distribution}

A QKD scheme is a protocol between two parties $A$ and $B$ that can perform quantum operations with the purpose of establishing a common classical string (their shared key), which afterwards they can use to communicate privately in a pre-agreed encryption scheme (such as a one-time-pad). Therefore, it is required that any third party, that might be eavesdropping, is not able to extract information about the key, thus compromising the privacy of the communication. The eavesdropper is usually called Eve ($E$). Bennett and Brassard~\cite{ben:bra:84} in 1984, and Ekert~\cite{eke:91} in 1991, proposed the first QKD protocols, upon which most of the discrete variables QKD protocols are based. Since then a lot of modifications and improvements have been proposed in order to achieve unconditionally secure and practical QKD schemes, by taking advantage of the laws of quantum mechanics~\cite{may:01,ren:05,sca:ren:08,tom:lev:17}. 

QKD protocols can be divided in two categories: the prepare-and-measure (PM) protocols and the entanglement-based (EB) protocols. In the former, the key is obtained by performing measurements on the quantum states that the parties exchange (ideally these states are pure), while in the latter, the parties share entangled states and they obtain the key by performing local measurements on their part of the entangled pair (these reduced states are mixed). In both kinds, some authenticated classical communication between the parties is also required for establishing the shared key. The aforementioned BB84~\cite{ben:bra:84} and E91~\cite{eke:91} protocols are the first, simplest and most illustrative representatives of PM and EB protocols, respectively. The security of these protocols is based on physical laws (as opposite to computational hardness assumptions for the current classical cryptosystems), specifically on principles and properties of quantum mechanics, such as the Heisenberg uncertainty principle, the non-cloning theorem~\cite{woo:zur:82}, the monogamy of entanglement~\cite{cof:kun:woo:00}, as well as the violation of Bell's inequalities~\cite{bel:64,chsh:ineq:69}. In 1992, Bennet, Brassard and Mermin~\cite{ben:bra:mer:92} showed that the security analysis of a simpler EB protocol similar to the E91, could be reduced to the security analysis of the BB84 protocol, thus creating a whole new context in quantum cryptography. Since then, the relationship between the presence of entanglement (and specifically the ability of the involved parties to certify or distil entanglement) and the security of QKD protocols has been thoroughly investigated~\cite{aci:mas:gis:03,cur:lew:lut:04,cur:guh:lew:lut:05,aci:gis:mas:06}.  
 


In this context, a quite common technique, when it comes to proving security of PM QKD protocols, is to consider an equivalent EB protocol and prove its security. Security of the latter implies security of the former. The proofs of security for EB protocols are widely based on various entropy inequalities that have been proposed~\cite{maa:uff:88,ren:boi:09,ber:chr:col:ren:ren:10,weh:win:10,tom:ren:11,ng:ber:weh:12,col:ber:tom:weh:17}, and they provide bounds on the maximum information that an eavesdropper $E$ can extract, depending on her attacks. These entropy bounds are then used to calculate the key rate that the parties $A$ and $B$ can securely obtain~\cite{dev:win:05,tom:lim:gis:ren:12}. 

In order to calculate the aforementioned entropy bounds one considers the probability distributions that result from the measurements that $A$ and $B$ perform on the pairs of the entangled states that they share, followed by error-correction and other post-processing techniques that they might choose to use. The statistics that $A$ and $B$ obtain depend crucially on the measurement devices that they possess; since they share entanglement their measurement data should be correlated, and if not, the parties conclude that their measurement devices cannot be trusted. This happens either because $E$ is interfering in the communication and introduces disturbance or because the measurement devices themselves are not properly working. If the measurement devices are not working properly, $E$ can use that to her advantage, thus compromising the security of the protocol. To overcome this problem, a whole new area of QKD has been created, the so-called device-independent QKD~\cite{pir:aci:bru:gis:mas:sca:09,mas:pir:aci:11,vaz:vid:14,agu:ram:kof:paw:16}. In this framework, the measurement devices of $A$ and $B$ are not trusted -- they are rather considered as black boxes that generate probability distributions, which do not necessarily result from measuring pairs of entangled states; they might even be classical probability distributions that $E$ is providing. Therefore, $A$ and $B$ have an extra task to certify the presence of entanglement, i.e., to make sure that their measurement data truly come from the entangled states that they assume to share. The correlations are tested through the violation of various Bell-type inequalities, depending on the dimension of the systems exchanged, the kind of the entangled states and the type of measurements they assume~\cite{cla:hor:shi:holt:69,col:gis:lin:mas:pop:02,sal:aug:rem:tur:wit:aci:pir:17}. We should note though that dealing with a device-independent scenario is much more complicated not only theoretically but also during the evaluation of the respective figures of merit. For this reason, the trusted device protocols have not been abandoned, as they are quite easier to deal with in practice. On the same time the technological progress promises higher trust in the devices in the future. An intermediate scenario exists, the so-called semi-device-independent QKD, in which one of the parties does not trust his device -- let's say $B$ -- while the other -- let's say $A$ -- does. In this case, the probability distribution of $B$ is assumed to be classical, while $A$'s probability distribution is assumed to come from a quantum state, resulting in a joint probability distribution of a so-called classical-quantum state. In this case, the correlations are tested through the violation of the respective steering inequalities~\cite{wis:jon:doh:07,jon:wis:doh:07,he:dru:rei:11,skr:cav:15,skr:cav:17,cav:he:rei:wis:11,he:rei:13,rei:dru:bow:cav:lam:bac:and:leu:09,bra:cav:wal:sca:wis:12}. Consequently, these three different levels of trust to the devices are connected and can be studied along with three basic features that characterise the entanglement properties of quantum states, namely their separability, locality and steerability, respectively. Here, we should also stress that the aforementioned equivalence between PM and EB protocols does not in general hold if the devices are not trusted. However, under specific assumptions about the devices we can have this equivalence in the device-independent case~\cite{woo:pir:15}.

Besides the problem of untrusted devices, there is another related issue that should be resolved if we really want to speak about unconditionally secure QKD protocols~\cite{beaudry:15}, and that is the randomness problem. Most of QKD protocols assume that the parties have access to some source of randomness and this assumption is crucial when it comes to proving security. As an example, one can consider the BB84 protocol, where $A$ and $B$  randomly choose the preparation and measurement bases, respectively, and this assumption ensures the security of the protocol. However, it is practically very hard to have perfect randomness sources, and $E$ can use this to her advantage. Therefore, the parties should be able to check how good their randomness sources are for the requirements of the protocol they want to execute; in other words, they should be able to certify randomness, such that $E$'s possible interference does not compromise the security of the protocol. On the same time, a lot of effort has been put in order to find ways that would enable access to more randomness. These issues of randomness certification and generation has been addressed in numerous studies~\cite{pir:aci:mas:gir:boy:mat:mau:olm:hay:luo:man:mon:10,aci:mas:pir:12,mat:skr:bra:cav:aci:15,aci:pir:ver:wit:16,aci:mas:16,and:bad:dum:cab:18}.

We conclude this section by mentioning that it has been shown in several studies~\cite{bec:tit:00,cer:bou:kar:gis:02,bru:chr:eke:eng:ber:kas:mac:03,nik:alb:05,she:sca:10,cha:15} that when the parties exchange higher dimensional systems (qudits instead of qubits), the respective QKD protocols can tolerate more noise than the 2-dimensional ones, thus opening a new direction in both the theoretical and practical investigation of QKD. Due to the high dimension of their positions space, the use of QWs in QKD seems to be a very promising option, as we will show in Chapter 3.


\section{One-dimensional discrete-time quantum walks}
\label{sec:preliminaries}
In the work presented in this thesis, we consider dicrete-time QWs (DTQWs) on the line and on the circle. In this section we will introduce and briefly describe some of their basic properties.
In a DTQW on the infinite line, we consider the movement of a walker along discrete positions on  it, labeled by $i \in \mathbb{Z}$. At each step the particle coherently moves to the left and to the right, depending on the state of an internal degree of freedom, the so-called coin state. 
The Hilbert space of the QW $H_{qw}$ is the tensor product of the position Hilbert space $H_p$ and the coin Hilbert space $H_c$, $H_{qw}=H_p\otimes H_c$. The position Hilbert space is $H_p=\text{span}\{\ket{i}, i\in \mathbb{Z}\}$, and the coin Hilbert space $H_c$ is spanned by the two possible coin states $\ket{R},\ket{L}$ corresponding to heads and tails. The letters $R$ and $L$ stand for right and left according to which side the walker is moving, when the coin shows heads or tails, respectively. 
A single step of the walk is given by the unitary evolution operator
$$\hat{U}_{qw}= \hat{S}\cdot (\hat{I}_p\otimes \hat{U}_c),$$
where $\hat{I}_p$ is the identity operator in $H_p$, $\hat{U}_c$ is a rotation in $H_c$ and $\hat{S}$ is the so-called shift operator, given by
$$\hat{S}=\sum_{i\in\mathbb{Z}} \ket{i+1}\bra{i}\otimes \ket{R}\bra{R}+ \ket{i-1}\bra{i}\otimes \ket{L}\bra{L},$$
and coherently moves the walker one position to the right and to the left on the line, depending on its coin state.\\
The general expression for $\hat{U}_c$ is:
\begin{equation}
\label{coin}
\hat{U}_c = \hat{U}_c(\theta, \xi, \zeta) = \left[
				\begin{array}{cc}
					e^{i\xi}
					\cos\theta & e^{i\zeta}
					\sin\theta\\
					-e^{-i\zeta}
					\sin\theta & e^{-i\xi}
					\cos\theta
				\end{array}
			\right].
\end{equation}
 
\subsection{Shift operator on the circle}
In the case of a DTQW on the circle, the walker hops along discrete positions on it.
To simulate such a walk, one could either identify the positions $-P$ and $P$ of a line, or connect the two, thus altering the corresponding shift operator. In the former, the circle has an even number of positions ($2P$), while in the latter it has an odd number of positions ($2P+1$).

In both cases, we can relabel the positions Hilbert space to be $\mathcal{H}_p =\span\{\ket{i}:i\in\{0,\dots,P-1\}\},$
and  write  the shift operator on the circle with $P$ positions as

\begin{eqnarray}
\label{shift}
\hat{S}& = &\sum_{i= 0}^{P-1} \Big(\ket{i+1 \ppmod {P}}\bra{i}  \otimes \ket{R} \bra{R} + \ket{i-1 \ppmod P}\bra{i}  \otimes \ket{L} \bra{L} \Big)\nonumber\\
		&=& \hat{T}_1 \otimes \ket{R} \bra{R} + \hat{T}_{-1} \otimes \ket{L} \bra{L},
\end{eqnarray}
where 
\begin{equation}
\hat{T}_x = \sum_{i=0}^{P-1}\ket{ i+x \ppmod P}\bra{i}
\end{equation}
is the translation operator for $x$ positions.



\section{Free-fermion systems exhibiting non-trivial topological order}
Topological phases of matter are a subject of active research during the last decades, as they constitute a whole new paradigm in condensed matter physics. Since the seminal paper of Haldane~\cite{hal:88}, where the anomalous Hall insulator was discovered, there has been an intense investigation of these ``exotic'' phases of matter~\cite{wen:90, kos:tho:72, kos:tho:73, hal:83}. In contrast to the well-studied ``standard'' quantum phases of matter, described by local order parameters (see for example Anderson's classification~\cite{and:07}), the ground states of topological systems are  characterised by global order parameters, which are called topological invariants, and they have the same value throughout the same topological phase. If a phase is topologically trivial, then the value of the associated invariant is 0. Examples of such invariants are the Chern numbers~\cite{tknn:82}, the Berry geometric phase~\cite{ber:84}, and non-local string parameters~\cite{zen:che:zho:wen:15} (cf.~\cite{ando:13}).  
 Free-fermion systems that describe insulators and superconductors with an energy gap can exhibit topological features, as long as their Hamiltonians possess some of the following symmetries, namely time-reversal (TRS), particle-hole (PHS) and chiral symmetry (CS). The respective topological phases can be classified according to these symmetries and the dimension of the system~\cite{kit:09, sch:ryu:fur:lud:08}. 
 
Due to the discrete nature of the topological invariants, Hamiltonians of gapped systems in different topological phases cannot be smoothly transformed from one into the other unless passing through a gap-vanishing region of criticality. In other words, for a topological phase transition (PT) to occur, the energy gap has to close. Topological PTs are quite different from the ``standard'' quantum PTs~\cite{sac:07}, which are traditionally described by the Landau theory~\cite{lan:37}. According to the Landau theory, a quantum PT occurs when we adiabatically change a parameter of the system around a region of criticality at temperature zero, thus breaking a local symmetry of the system. On the other hand, when it comes to topological PTs, there is no symmetry breaking; the symmetries of the system are rather preserved.
  
 
At zero-temperature, the critical behaviour of systems featuring topological order is accompanied by the existence of edge states at the boundary between two distinct topological phases. These edge states are symmetry-protected, i.e., they are robust against perturbations of the Hamiltonian that preserve the respective symmetries, and their presence is predicted by the bulk-to-boundary correspondence principle~\cite{x:g:wen:91,ryu:hat:02}. 


In addition to the standard local symmetry breaking and the aforementioned global topological order parameters, several information-theoretic quantities were used to study PTs in general, such as entanglement measures~\cite{vid:lat:ric:kit:03,ham:ion:zan:05,ham:zha:haa:lid:08,hal:ham:12} and the fidelity~\cite{aba:ham:zan:08,zan:pau:06,zan:ven:gio:07,zha:zho:09, oli:sac:14, pau:sac:nog:vie:dug,gu:kwo:nin:wen:lin:08,maz:ham:12}. Whenever there is a PT, the state of the system changes significantly and the fidelity, as a measure of the distinguishability between two quantum states, signals out this change. In~\cite{zan:gio:coz:07} the authors analysed the intimate connection between the pure-state fidelity and the Berry phase, showing that the fidelity-induced Riemannian metric and the Berry curvature are the real and imaginary part, respectively, of the so-called quantum geometric tensor and thus, they provide a universal framework for the study of quantum PTs.

A question that naturally arises is whether there is any kind of topological order at finite temperatures  and which could be the appropriate quantities (``topological order parameters'') to describe possible PTs. In the context of systems in thermal equilibrium, several different approaches have been used to tackle this problem~\cite{viy:riv:del:12,riv:viy:del:13,bar:bar:krau:rico:ima:zol:die:13,nie:hub:14,bar:waw:alt:flei:die:17,grus:17} and various types of mixed-state generalisations of geometric phases~\cite{uhl:89,sjo:15} were used to infer topological phase transitions of systems at finite temperature (for the pure-state case of the Berry phase, see~\cite{car:pac:05,reu:har:ple:07}). The most promising one is based on the work of Uhlmann~\cite{uhl:89}, who extended the notion of geometrical phases from pure states to density matrices. The concept of the Uhlmann holonomy, and certain quantities that can be derived from it, were used to infer PTs at finite temperatures~\cite{hub:93,viy:riv:del:14, viy:riv:del:2d:14,zho:aro:14,pau:vie:08,viy:riv:mar:15}. There exist several proposals for the observation of the Uhlmann geometric phase~\cite{tid:sjo:03, abe:kul:sjo:oi:07} and experimental demonstrations have been reported in~\cite{uhl:pha:exp:11,viy:riv:gas:wal:fil:del:18}. Nevertheless, the physical meaning of these quantities and their relevance to the observable properties of the corresponding systems stay as an interesting open question~\cite{sjo:15, kem:que:smi:16, bud:die:15}. On the other hand, the fidelity is, through the Bures metric~\cite{zan:ven:gio:07}, closely related to the Uhlmann connection. Therefore, both the fidelity and the Uhlmann connection can be used to infer the possibility of PTs, as shown in~\cite{pau:vie:08} for the case of the BCS superconductivity.


\section{The fidelity, the Uhlmann connection and their use in the study of phase transitions of systems in equilibrium}
In Chapters 4 and 5, we present our study of PTs at finite temperatures for topological systems in equilibrium, where we will use the well-established fidelity approach, as well as a quantity associated to the Uhlmann connection (along the lines of~\cite{pau:vie:08}). In this section we provide a general overview of the fidelity and its relationship with the Uhlmann connection, and how they can be used in order to infer the existence of PTs at finite temperatures.

We start by considering the classical fidelity between two probability distributions. Given the probability distributions $\{p_x\}_{x\in L}$ and $\{q_x\}_{x\in L}$ of two random variables $X$ and $Y$, respectively, ranging over the same set $L=\{1,2,\cdots,n\}$, the classical fidelity is defined as
 \begin{equation*}
 	F(p_x,q_x)=\sum_{x\in L}\sqrt{p_x q_x}.
 \end{equation*}
It is a distinguishability measure for classical information, as it describes how ``close'' the two distributions are. Geometrically, it is the inner product between two (normalised) vectors $(\sqrt{p_1},\cdots,\sqrt{p_n})$ and $(\sqrt{q_1},\cdots,\sqrt{q_n})$ in the Euclidean space. 

We now proceed to introduce the quantum analog of classical fidelity, the fidelity between two quantum states $\rho_1$ and $\rho_2$. One way to do so is to take into account that each time we measure an observable, let's say $\hat{O}$, with respect to two quantum states $\rho_1$ and $\rho_2$, we end up with two classical probability distributions $\{p_x^{\hat{O}}\}_{x\in L}$ and $\{q_x^{\hat{O}}\}_{x\in L}$, respectively, for which we can calculate the classical fidelity $F(p_x^{\hat{O}},q_x^{\hat{O}})$. It has been shown that the quantum fidelity given as 

\begin{equation*}
	F(\rho_1,\rho_2)=\tr\sqrt{\sqrt{\rho_1}\rho_2\sqrt{\rho_1}},
\end{equation*}
is always smaller or equal than the classical fidelity between the two respective probability distributions
$$F(\rho_1,\rho_2)\leq F(p_x^{\hat{O}},q_x^{\hat{O}}).$$
The equality is achieved for a specific positive-operator valued measurement (POVM) associated to a so-called optimal observable for the two states, see~\cite{fuchs:96}.

In analogy to the classical case, $F(\rho_1,\rho_2)$ is associated to the distance between $\rho_1$ and $\rho_2$, thus it is a measure of their distinguishability. Nevertheless, we should stress that fidelity itself is not a distance, however it can be associated to the Bures distance in the space of density matrices, as follows:

\begin{equation*}
	d_B(\rho_1,\rho_2)=\sqrt{2(1-F(\rho_1,\rho_2))},
\end{equation*}
with $d_B(\rho_1,\rho_2)$ being the Bures distance.
The fidelity takes its minimum value $F(\rho_1,\rho_2)=0$, when the two states are completely distinguishable and this corresponds to the maximum value of the Bures distance $d_B(\rho_1,\rho_2)=1$.
On the other hand, the Bures distance is minimum $d_B(\rho_1,\rho_2)=0$, when the fidelity achieves its maximum value $F(\rho_1,\rho_2)=1$, which means that the states $\rho_1$ and $\rho_2$ are completely indistinguishable.

A few basic properties of the fidelity are the following:

\begin{enumerate}
	\item Fidelity is symmetric with respect to its arguments: $F(\rho_1,\rho_2)=F(\rho_2,\rho_1)$.
	\item Fidelity is invariant under unitary transformations: $F(U\rho_1 U^\dagger,U\rho_2 U^\dagger)=F(\rho_1,\rho_2).$
\end{enumerate}
In the special case that one of the two states is pure, for example if $\rho_2=\ket{\psi}\bra{\psi}$, the fidelity is $F(\ket{\psi},\rho_1)=\sqrt{\braket{\psi|\rho_1|\psi}}$. Furthermore, for two pure states $\ket{\psi}$ and $\ket{\phi}$: $F(\ket{\psi},\ket{\phi})=|\braket{\psi|\phi}|,$ which is exactly the overlap between the two states.

We proceed by showing the relationship between the fidelity and the Uhlmann connection. The set of mixed states is convex but not linear in general, i.e., for any two mixed states $\rho_1$ and $\rho_2$ and scalars $\lambda_1$ and $\lambda_2$ the linear combination $\lambda_1\rho_1+\lambda_2\rho_2$ is not necessarily a mixed state. Nevertheless, a convex combination of $\rho_1$ and $\rho_2$ is a mixed state, i.e, for
 $\lambda_1,\lambda_2\geq 0$ and $\lambda_1+\lambda_2=1$, the linear combination $\lambda_1\rho_1+\lambda_2\rho_2$ belongs in the set of mixed states. This feature of non-linearity imposes significant restrictions when it comes to perform a geometric study. 
 On the other hand, we do not have this issue in the case of pure states, since a pure state $\rho=\ket{\psi}\bra{\psi}$, can be treated as a projection on the subspace spanned by $\ket{\psi}$, therefore inheriting the geometric properties of the Hilbert space. Notice the $U(1)-$~gauge freedom; $\ket{\psi}$ and $e^{i\phi}\ket{\psi}$ correspond to the same state $\rho=\ket{\psi}\bra{\psi}$.

To overcome this restrictions, one can introduce the concept of the purification of a mixed state, that is any mixed state in a certain Hilbert space can be seen as the reduced state of a pure state in a different (larger) Hilbert space. Specifically, let us assume that we have a mixed state $\rho$ represented by the corresponding density matrix acting on a finite-dimensional Hilbert space $H_1$. Then we can always consider a second Hilbert space $H_2$ and a pure state $\ket{\Psi}\in H_1\otimes H_2$, such that $$\rho=\tr_2 (\ket{\Psi}\bra{\Psi}).$$ The state $\ket{\Psi}$ is called a purification of $\rho$. We should stress that for a state $\rho$ one can find in general more than one purifications. Nevertheless, there is a choice of purification of particular interest, the one that reveals the relationship between the fidelity and the Uhlmann connection. This purification belongs in the so-called Hilbert-Schmidt space: 
 \begin{definition}
 	(Hilbert-Schmidt space)\\Given a finite-dimensional Hilbert space $H$, the corresponding Hilbert-Schmidt space is the tensor product $H\otimes H^*$, where $H^*$ is the dual of $H$. This space is equipped with the respective Hilbert-Schmidt inner product, which for $w_1,w_2\in H\otimes H^*$ is defined as
 	$$\langle w_1,w_2\rangle= \tr (w_1^{\dagger}w_2).$$ 
 \end{definition}  
Any mixed state $\rho\in H$ can be purified by means of $w\in H\otimes H^*$ as $\rho=ww^\dagger$, where $w$ is called the amplitude of $\rho$ in this case. Notice that there is a $U(n)-$~gauge freedom in the choice of the amplitude, analogously to the $U(1)-$~gauge freedom in the case of pure states; $w$ and $wU$ with $U$ being unitary correspond to the same state $\rho=ww^\dagger$. The choice of the Hilbert-Schmidt space is quite special, since there is no space smaller than $H\otimes H^*$ where we can purify $\rho$, while choosing a larger space is not necessary because the purified density matrices will always belong in a subspace isomorphic to the Hilbert-Schmidt space~\cite{ben:zyc:06}. In what follows, we describe how a particular choice of the amplitude reveals the relationship between the fidelity and the Uhlmann connection.

Two amplitudes $w_{1}$ and $w_{2}$, such that $\rho_{1}=w_1 w_1^\dagger$ and $\rho_{2}=w_2 w_2^\dagger$, are said to be parallel in the Uhlmann sense if they minimise the distance induced by the Hilbert-Schmidt inner product $\langle w_{2},w_{1}\rangle=\tr(w_{2}^{\dagger}w_{1})$: 
\begin{equation*}
||w_2-w_1||^2=\tr(w_2-w_1)^\dagger (w_2-w_1)=2(1-\text{Re}\langle w_2,w_1\rangle).	
\end{equation*}
Minimising $||w_2-w_1||^2$ is equivalent to maximising $\text{Re}\langle w_{2},w_{1}\rangle$: 
\begin{equation*}
\text{Re}\langle w_{2},w_{1}\rangle\leq|\langle w_{2},w_{1}\rangle|  =|\tr(w_{2}^{\dagger}w_1)|.
\end{equation*}
Considering the polar decompositions $w_{i}=\sqrt{\rho_{i}}U_{i},i\in\{1,2\}$, where the $U_i$'s are unitary matrices, the above inequality becomes
\begin{equation*}
\text{Re}\langle w_{2},w_{1}\rangle \leq 
|\tr(U_{2}^{\dagger}\sqrt{\rho_{2}}\sqrt{\rho_{1}}U_{1})|.
\end{equation*}
Using the polar decomposition $\sqrt{\rho_{2}}\sqrt{\rho_{1}}=|\sqrt{\rho_{2}}\sqrt{\rho_{1}}|V$, with $V$ unitary, and the cyclic property of the trace, we obtain
\begin{equation*}
\text{Re}\langle w_{2},w_{1}\rangle \leq 
|\tr(|\sqrt{\rho_{2}}\sqrt{\rho_{1}}  |VU_{1}U_{2}^{\dagger})|.
\end{equation*}
The Cauchy-Schwarz inequality implies 
\begin{equation*}
\text{Re}\langle w_{2},w_{1}\rangle \leq 
\tr|\sqrt{\rho_{2}}\sqrt{\rho_{1}}|,
\end{equation*}
with the equality holding for $V U_1 U_2^\dagger=I$, where $I$ is the identity. Finally, we can write $|\sqrt{\rho_{2}}\sqrt{\rho_{1}}|=\sqrt{(\sqrt{\rho_{2}}\sqrt{\rho_{1}})^{\dagger}(\sqrt{\rho_{2}}\sqrt{\rho_{1}})}$ and get
\begin{equation*}
\text{Re}\langle w_{2},w_{1}\rangle \leq 
\tr\sqrt{(\sqrt{\rho_{1}}\rho_{2}  \sqrt{\rho_{1}})}=F(\rho_{1},\rho_{2}),
\end{equation*}
which shows the relationship between the fidelity and the Uhlmann connection, as characterised by $V$, the so-called Uhlmann factor (associated to the choice of gauge for the amplitudes).

To illustrate this relationship better, let us consider the Uhlmann parallel transport condition. Suppose that $\rho(t)$ is a closed curve of density matrices parametrised by $t \in [0,1]$. Then, given an initial state $\rho(0)$ and the corresponding amplitude $w(0)$, the Uhlmann parallel transport condition, taken for an infinitesimal time period $\delta t$, yields a unique curve in the space of amplitudes (called the horizontal lift), along which $w(t)$ and $w(t+\delta t)$ are parallel, $\forall t$. The length of the curve in the space of amplitudes -- with respect to the metric induced by the Hilbert-Schmidt inner product -- is equal to the length of the corresponding curve in the space of density matrices -- with respect to the Bures metric. The respective Bures distance is given in terms of the fidelity as $d_B(\rho_1,\rho_2)=\sqrt{2(1-F(\rho_1,\rho_2))}$.

Finally, we present how we can apply the above concepts in the study of PTs. Consider two close points $t$ and $t+\delta t$ in the parameter space and  the respective states $\rho(t)$ and $\rho(t+\delta t)$ in the space of density matrices. If the two states belong to the same phase, they almost commute, hence $V\approx I$ and $\sqrt{\rho(t+\delta t)}\sqrt{\rho(t)}\approx|\sqrt{\rho(t+\delta t)}\sqrt{\rho(t)}|$. Moreover, since they are almost indistinguishable, $F(\rho(t),\rho(t+\delta t))\approx 1$. On the other hand, if $\rho(t)$ and $\rho(t+\delta t)$ belong to different phases, they must be significantly different, thus their fidelity must be smaller than one~\cite{zan:pau:06}. The difference between the two states can be in their spectra or their eigenbases. In the case of the latter, we also have non-trivial $V\neq I$~\cite{pau:vie:08}.

To quantify the difference between the Uhlmann factor $V$ and the identity, we will use the following quantity, as defined in~\cite{pau:vie:08}:

\begin{equation} 
\Delta(\rho(t),\rho(t+\delta t)):=  F(\rho(t),\rho(t+\delta t))- \tr(\sqrt{\rho(t+\delta t)}\sqrt{\rho(t)}).
\end{equation}
For $\rho(t)$ and $\rho(t+\delta t)$ from the same phase, $F(\rho(t),\rho(t+\delta t))=\tr(\sqrt{\rho(t+\delta t)}\sqrt{\rho(t)})\approx 1$, therefore $\Delta(\rho(t),\rho(t+\delta t))\approx 0$, while for $\rho(t)$ and $\rho(t+\delta t)$ from different phases, $F(\rho(t),\rho(t+\delta t))\neq 1$ and in the case the Uhlmann factor is also non-trivial, we have $\Delta(\rho(t),\rho(t+\delta t))\neq0$.

Summarising the above, the departure of fidelity from 1 and the departure of $\Delta$ from 0 are signalling the PT points. 


\section{Phase transitions of systems out of equilibrium }
The real time evolution of closed quantum systems out of equilibrium has some surprising similarities with thermal PTs, as noticed by Heyl, Polkovnikov and Kehrein~\cite{hey:pol:keh:13}. They coined the term Dynamical Quantum PTs (DQPTs) to describe the non-analytic behaviour of certain dynamical observables after a sudden quench in one of the parameters of the Hamiltonian. Since then, the study of DQPTs became an active field of research and a lot of progress has been achieved in comparing and connecting them to the equilibrium PTs~\cite{kar:sch:13,and:sir:14,vaj:dor:14,hey:15,kar:sch:17,hal:zau:17,zau:hal:17,hom:abe:zau:hal:17}. Along those lines, there exist several studies of DQPTs for systems featuring non-trivial topological properties~\cite{vaj:dor:15,sch:keh:15,bud:hey:16,hua:bal:16,jaf:joh:17,sed:flei:sir:17}. DQPTs have been experimentally observed in systems of trapped ions~\cite{jur:etal:17} and systems of cold atoms in optical lattices, that both exhibit topological features~\cite{fla:etal:18}. The figure of merit in the study of DQPTs is the Loschmidt Echo (LE) and its derivatives, which have been extensively used in the analysis of quantum criticality~\cite{qua:son:liu:zan:sun:06,zan:pau:06,zan:qua:wan:sun:07,pol:muk:gre:moo:10,jac:ven:zan:11} and quantum quenches~\cite{ven:zan:10}. At finite temperature, generalisations of the zero-temperature LE were proposed, based on the mixed-state Uhlmann fidelity~\cite{ven:tac:san:san:11,jac:ven:zan:11}, and the interferometric mixed-state geometric phase~\cite{hey:bud:17, bah:ban:dut:17}.
 For alternative approaches to finite-temperature DPTs, see~\cite{lan:fra:hal:17,hal:zau:mcc:veg:sch:kas:17}. 
 As already stressed, the fidelity has been employed numerous times in the study of PTs~\cite{zan:pau:06,pau:sac:nog:vie:dug,zan:ven:gio:07,aba:ham:zan:08,zha:zho:09}, while the interferometric mixed-state geometric phase was introduced in~\cite{sjo:pat:eke:ana:eri:oi:ved:00}. The two quantities are in general different and it comes as no surprise that they give different predictions for the finite temperature behaviour of systems with topological order~\cite{bah:ban:dut:17}: the fidelity LE does not show DPTs at finite temperatures, while the interferometric LE indicates their persistence. In Chapter 6 we study DPTs of topological systems and clarify what their fate at finite temperature truly is, and which of the two opposite predictions better captures their many-body nature. 

\section{Structure of the thesis}
The work presented this thesis led to the publication of five papers; four of the them are already published in peer-reviewed journals and one is submitted to a peer-reviewed journal and is currently under review (it is also available as a preprint in arXiv):
\begin{itemize}
	\item C. Vlachou, J. Rodrigues, P. Mateus, N. Paunkovi\'c, and A. Souto, Quantum walk public-key cryptographic system, International Journal of Quantum Information, 13(7):1550050, 2015. 
	\item C. Vlachou, W. Krawec, P. Mateus, N. Paunkovi\'c, and A. Souto, Quantum key distribution with quantum walks, arXiv:1710.07979, (2017).
		\item B. Mera, C. Vlachou, N. Paunkovi\'c, and V. R. Vieira, Uhlmann connection in fermionic systems undergoing phase transitions, Phys. Rev. Lett., 119:015702, 2017.
		\item B. Mera, C. Vlachou, N. Paunkovi\'c, and V. R. Vieira, Boltzmann-Gibbs states in topological quantum walks and associated many-body systems: fidelity and Uhlmann parallel transport analysis of phase transitions, Journal of Physics A: Mathematical and Theoretical, 50(36):365302, 2017.
	\item B. Mera, C. Vlachou, N. Paunkovi\'c, V. R. Vieira, and O. Viyuela, Dynamical phase transitions at finite temperature from fidelity and interferometric Loschmidt echo induced metrics, Phys. Rev. B, 97:094110, 2018.
\end{itemize}

The thesis begins with the Introduction in Chapter 1, which puts the developed work into context.
In Part I of the thesis, which includes Chapters 2 and 3, we present our work on the applications of QWs in cryptography. In particular, in Chapter 2 we propose a quantum public-key cryptosystem in which the public key is generated by performing a QW. We show that the protocol is secure and we analyse the complexity of the public-key generation and the encryption/ decryption procedures.
In Chapter 3, we take advantage of the properties of QWs to design new secure QKD schemes. In particular, we propose three new QKD protocols; a two-way protocol, a one-way protocol of the BB84 type, and a semi-quantum protocol. We prove the security of the first two and the robustness against eavesdropping of the third. For all the aforementioned QKD protocols, we describe in detail the quantum memory requirements for all the parties (legitimate and eavesdropper).

In Part II of the thesis, which includes Chapters 4, 5 and 6, we present our study of PTs at finite temperatures for systems in and out of equilibrium that exhibit non-trivial topological features. In particular, in Chapter 4 we study the behaviour of the fidelity and the Uhlmann connection in systems of fermions undergoing PTs, both topologically trivial and non-trivial. By means of this approach, we show the absence of thermally driven PTs in the case of topological insulators and superconductors. Furthermore, by studying their edge states, we confirm the results obtained by the fidelity and the Uhlmann connection study, that is the gradual disappearance of the topological features at finite temperatures. We also clarify what is the relevant parameter space associated with the Uhlmann connection so that it signals the existence of topological order in mixed states. Among others, we studied the behaviour with respect to temperature of the Majorana modes (edge states of topological superconductors) which are basic constituents for achieving quantum memories~\cite{ali:12,ipp:riz:gio:maz:16,majorana:17,majorana:twist:17}.  
In Chapter 5, we consider QW protocols that are known to simulate topological phases and the respective quantum PTs~\cite{kit:rud:ber:dem:10,kit:12} for chiral symmetric Hamiltonians, in order to investigate whether the topological order of these systems at zero temperature is also maintained at finite temperatures. 
Using the same approach as in Chapter 4, we conclude that no temperature-driven PTs occur, i.e., the topological behaviour is washed out gradually as temperature increases. However, we find finite-temperature parameter-driven PTs. In Chapter 6, we move to topological systems out-of-equilibrium and we study finite-temperature DQPTs by means of the fidelity and the interferometric LE induced metrics. When generalising the associated dynamical susceptibilities -- which coincide at zero temperature -- at finite temperatures one finds that they behave very differently: using the fidelity LE, the zero temperature DQPTs are gradually washed away with temperature, while the interferometric counterpart exhibits finite-temperature DPTs. We analyse the physical differences between the two, and argue which is the more suitable quantity to study, when it comes to perform relevant experiments in topological many-body systems.

Finally, in Chapter 7 we summarise all the above and present our conclusions.

%\bibliography{bibforthesis}
%\bibliographystyle{unsrt}
%\end{document}
