%\documentclass[11pt]{report}



%\begin{document}

\chapter{Quantum Oblivious Transfer}
%\addcontentsline{toc}{chapter}{Introduction}


In a recent survey on classical oblivious transfer (OT) \cite{YAVV22}, all the analysed protocols require some form of asymmetric cryptography. Indeed, in the classical setting, it is impossible to develop information-theoretic secure OT or even reduce it to one-way functions, requiring some public-key computational assumptions. As shown by Impaggliazzo and Rudich \cite{IR89}, one-way functions (symmetric cryptography) alone do not imply key agreement (asymmetric cryptography). Also, Gertner et al. \cite{GKMRV00} pointed out that since it is known that OT implies key agreement, this sets a separation between symmetric cryptography and OT, leading to the conclusion that OT cannot be generated alone by symmetric cryptography. Otherwise, one could use one-way functions to implement key agreement through the OT construction. This poses a threat to all classical OT protocols \cite{EGL85, NP01, CO15} that are based on mathematical assumptions provably broken by a quantum computer \cite{Sho95}. Besides the security problem, asymmetric cryptography tends to be computationally more complex than symmetric cryptography, creating a problem in terms of speed when a large number of OTs are required. The classical post-quantum approach, thrives to find protocols resistant against quantum computer attacks. However, these are still based on complexity problems and are not necessarily less computationally expensive, than the previously mentioned ones. 

In parallel to the classical post-quantum approach, the quantum cryptography community presented some OT protocols based on quantum technologies to tackle this security issue. Intriguingly enough, more than a decade before the first classical OT by Rabin (1981, \cite{Rabin81}) was published, Wiesner proposed a similar concept. However, at the time it was rejected for publication due to the lack of acceptance in the research community. In fact, the first published quantum OT (QOT) protocol, known as the BBCS (Bennett-Brassard-Cr{\'e}peau-Skubiszewska) protocol \cite{BBCS92} was only presented in 1992. Remarkably, there is a distinctive difference between classical and quantum OT from a security standpoint, as the latter is proved to be possible assuming only the existence of quantum-hard one-way functions \cite{GLSV21, BCKM21}. This means quantum OT requires weaker security assumptions than classical OT.

In this chapter, we review the particular topic of quantum OT. We mainly comment on several important OT protocols, their underlying security models and assumptions. To the best of our knowledge, there is no prior survey dedicated to quantum OT protocols alone. Usually, its analysis is integrated into more general surveys under the topic of "quantum cryptography", leading to a less in-depth exposition of the topic. For reference, we provide some distinctive reviews on the general topic of quantum cryptography \cite{BC96, B05, M06, F10, B15, PAB+20, PR21, SH22}.

(Falta fazer um resumos!)


%********************************** %First Section  **************************************
\section{Impossibility results}




%\bibliography{bibforthesis}
%\bibliographystyle{unsrt}
%\end{document}