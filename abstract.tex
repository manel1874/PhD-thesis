
\chapter*{Abstract}
\addcontentsline{toc}{section}{Abstract}
\noindent Quantum cryptography is an active area of research, since it was shown that the security of some of the classical cryptosystems currently used can be compromised by adversaries with access to quantum computers. In the first part of this thesis, we propose new secure quantum cryptosystems based on quantum walks, which have been proved very useful in several quantum computation tasks. In particular, we design a secure quantum public-key cryptosystem, as an alternative to the classical counterparts, whose security can be jeopardised by quantum adversaries. Moreover, we propose three new quantum key distribution protocols and analyse their security and robustness properties.

The design of long-term stable quantum memories is currently one of the biggest technological challenges towards the construction of operational and scalable quantum computers, and their existence or absence has serious implications in both classical and quantum cryptography. In the second part of this thesis, we study the finite-temperature behaviour of systems exhibiting topological order, as they possess some unique properties which could be used to construct quantum memories. By means of the fidelity between two quantum states and the Uhlmann parallel transport condition in the space of the purifications of density matrices, we investigate the existence of topological phase transitions at finite temperatures. We also probe the robustness of the edge states on the boundary between two different symmetry-protected topological phases with respect to the temperature. This analysis shows that there exist no thermal phase transitions and the topological features, present at zero temperature, are gradually smeared out as the temperature increases. Our study of the Majorana modes (edge states of topological superconductors) at low but finite temperatures, suggests that they could be used to achieve realistic quantum memories.

We also performed the same analysis for the effective Hamiltonians resulting from quantum walk protocols that have been shown to simulate symmetry-protected topological phases. The results that we obtained are consistent with the previous, indicating the absence of thermally-driven phase transitions. Moreover, in this case, our analysis revealed the existence of finite-temperature parameter-driven phase transitions.

Finally, we investigated the existence of finite-temperature phase transitions in topological systems out of equilibrium. So far, there exist two different approaches to infer the possibility of phase transitions  at finite temperatures for such systems, which were giving opposite predictions. We analytically derived the relevant quantities and showed the origin of such different behaviours. Moreover, we argued which is the most suitable approach that better captures the many-body nature of topological systems and can be used in realistic implementations.

\vfill

\begin{flushleft}
\textbf{Key-words:} quantum cryptography, quantum walks, quantum memories, topological phase transitions,  edge states
\end{flushleft}
