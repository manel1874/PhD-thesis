%\documentclass[11pt]{report}



%\begin{document}

\chapter{Conclusion}
\label{ch:conclusion}

In this thesis, the focus is on exploring and enhancing the utilization of quantum cryptography in secure multiparty computation (SMC) systems. 

Chapter~\ref{chapter_QOT} provides an overview of quantum oblivious transfer (QOT) protocols. Our analysis is centered around the use of oblivious keys, which facilitate the modular execution of secure multiparty computation (SMC) protocols by allowing the separation of quantum technology and secure computation. We also examine the threat posed by quantum hacking techniques and provide an evaluation of both practical and theoretical measures to mitigate these attacks.

In Chapter~\ref{classical-and-quantum-OT}, we conducted a theoretical comparison of the complexity of quantum and classical OT protocols to assess their impact on the efficiency of SMC protocols. This is motivated by the close connection between the Yao garbled circuit protocol and OT. We proposed an optimized version ($\Pi^{\textbf{BBCS}}_{\textbf{O}}$) of the BBCS-based QOT protocols and compared its transfer phase with that of the fastest known classical OT implementation, ALSZ13 \cite{ALSZ13}. Our conclusion was that the transfer phase of $\Pi^{\textbf{BBCS}}_{\textbf{O}}$ has the potential to be faster than that of the ALSZ13 OT extension while maintaining a much higher level of security. In contrast, the ALSZ13 protocol is only proven to be secure in the semi-honest model, whereas $\Pi^{\textbf{BBCS}}_{\textbf{O}}$ is secure in the malicious setting.
Furthermore, we compared the transfer phase of maliciously secure classical protocols, ALSZ15 \cite{ALSZ15} and KOS15 \cite{KOS15}, with that of $\Pi^{\textbf{BBCS}}_{\textbf{O}}$ and found that they have a greater computation and communication complexity than $\Pi^{\textbf{BBCS}}_{\textbf{O}}$.

In Chapter~\ref{ch:phylogenetic-trees}, we bring theory closer to practice by presenting a SMC protocol that uses quantum technologies to analyse distance-based algorithms of phylogenetic trees. Our proposed system integrates the use of ready-to-use libraries, such as CBMC-GC, Libscapi, and PHYLIP, to provide a complete, quantum-resistant solution. We implement and compare the performance of both a classical-only and a quantum-assisted system using simulated symmetric and oblivious keys. %The results from Chapter~\ref{classical-and-quantum-OT} suggest that the quantum-assisted version could have comparable efficiency to the classical-only version. However, this assumes that the overhead created by the oblivious key management system is not taken into account. Despite this difference, the quantum-assisted system provides a significantly higher level of security against quantum computer attacks.

In Chapter~\ref{ch:QOLE}, we presented a two-phase quantum oblivious linear evaluation (QOLE) protocol that, to the best of our knowledge, is the first quantum protocol proposed for this primitive. We establish its security within the quantum-UC framework, under the assumption of ideal commitments, thereby rendering it amenable to composition in a versatile manner. Furthermore, we present two generalisations of our protocol: the first attains QOLE in Galois fields of prime-power dimensions, while the second realizes quantum vector OLE. It's noteworthy that our protocol achieves everlasting security, ensuring information-theoretical resilience even in the face of potential future empowerment of dishonest parties after its execution.

\section{Future work}

Our findings outlined above have implications for both theoretical and practical areas of research. In terms of practical implementation, one can strive for a more efficient oblivious key management system. %Currently, the implementation utilized a file system to store the oblivious keys, which may not be the most efficient method. However, it was chosen for its modularity.
In the future, it would be beneficial to extend the implementation of Libscapi to include a method to directly access the oblivious keys stored in memory, potentially increasing efficiency.

In the theoretical realm, the work developed for protocol $\pi_{\textbf{QOLE}}$ can be expanded as well. Currently, its security has been analysed in the absence of noise. Proving security in the presence of noise would follow a similar approach (as seen in \cite{DFLSS09}). In case of noise, during Step 4 of the quantum phase of $\pi_{\textbf{QOLE}}$ (depicted in Figure~\ref{fig:fullprotocol}), Alice should abort the protocol if the error measurement ($err$) exceeds a pre-defined value $\nu$ attributed to the noise. This results in $err = \nu + \zeta '$, where $\zeta '$ represents the potential dishonest behavior of Bob. This adjustment reduces the lower bound of the min-entropy of Alice's functions $\mathbf{F}$ given Bob's side information, i.e.
$$H_{\min}(\mathbf{F} | \mathbf{Y} E)_{\sigma_{\mathbf{F}\mathbf{Y} E}} \geq \frac{n\log d}{2}\left(1 - h_d(\nu + \zeta ' + \zeta)\right).$$
While it is possible to extend the security of the protocol to include noise in the quantum states, its correctness cannot be guaranteed in such scenarios. Therefore, as future work, new protocols should be developed that account for noise and examine its impact on the security properties. As an initial suggestion, one could build upon existing work. The error-tolerant OLE combiner from \cite{PW08} provides a way to integrate several possibly faulty OLE instances into one correct OLE. Although this protocol ensures the correctness of the protocol under noise, it does not include a privacy-amplification phase robust against quantum side information. To bridge this gap, one could strive to apply a linear strong extractor to each OLE instance \cite{DPV+12}. For example, by considering a prime-power dimension, $d^M$, and the natural correspondence between $\mathbb{Z}_{d^M}$ and $(\mathbb{Z}_d)^M$, one could use the inner product seeded extractor in $(\mathbb{Z}_d)^M$ \cite{KK10}.


The $\pi_{\textbf{QOLE}}$ protocol can also be adapted to operate within the bounded-quantum-storage model, while preserving its security. In this adaptation, the test phase of $\pi^n_{\textbf{RWOLE}}$ is replaced by a waiting time $\Delta t$, limiting the amount of qudits that Bob can store. Further investigation into the impact of different noisy channels on the security properties of the protocol would be valuable. Additionally, to ensure the protocol's composability, an analysis within the bounded-quantum-storage-UC model, as proposed by Unruh \cite{U11}, should be performed.

Our protocol is a two-way scheme, in which Bob prepares and sends a quantum state, Alice performs an operation on it, and returns it to Bob, who finally measures the final state. There are several two-way QOT protocols in the literature \cite{CZK18, ASRP21, KST21, CKS10, CGS16}, with the one proposed by Amiri et al. \cite{ASRP21} demonstrating their experimental feasibility. This drives the motivation to further develop practical implementations of our protocol. Moreover, the security of our protocol can be enhanced by making it device-independent. One can look to the work of Kundu et al. \cite{KST21} as inspiration, who built upon the work of Chailloux et al. \cite{CKS10}. While the above-mentioned works primarily focus on two-way QOT protocols, recent studies have also proposed one-way, non-interactive protocols for device-independent \cite{BY21} and XOR QOT \cite{SSH+21}.

Finally, based on our results, one could construct quantum protocols for oblivious polynomial evaluation, which -- as mentioned in the beginning of Chapter~\ref{ch:QOLE} -- is another important primitive facilitating various applications.






%\bibliography{bibforthesis}
%\bibliographystyle{unsrt}
%\end{document}
