%\documentclass[11pt]{report}



%\begin{document}

\chapter{Introduction}
%\addcontentsline{toc}{chapter}{Introduction}

The field of data mining and data analysis has seen significant advancements with the increasing power of computers \cite{Wang11}. However, the need for large-scale data collection can lead to the compromise of sensitive and private information, particularly in fields such as genomics \cite{NS08, L02, Homer08, Gymrek13}. As a result, the aggregation of data from different sources is often restricted by laws and regulations such as the General Data Protection Regulation (GDPR) \cite{EUdataregulations2016}. While these regulations aim to protect individuals' privacy, they also limit the ability of honest parties to access the data needed to address important societal issues.

%********************************** %First Section  **************************************
\section*{Secure multiparty computation}


To address the privacy concerns outlined above, various privacy-enhancing technologies have been proposed, such as secure multiparty computation (SMC) \cite{Li2016, Armknecht2015, Yao82}. SMC enables a group of $n$ parties, denoted as $P_i$, to jointly compute a function $f( x_1, ..., x_n ) = (y_1, ..., y_n)$ without revealing their inputs to the other parties. The security requirements of SMC are equivalent to an ideal scenario, where each party $P_i$ sends their inputs to a separate, trusted third party, who then computes $f()$ and returns the corresponding output to each party.

The research area of SMC has been greatly advanced with the development of various protocols and frameworks \cite{Goldreich87, Bendlin11, D12}. These protocols can be broadly classified into two categories: those based on Boolean logic and those based on arithmetic logic. The efficiency and security of SMC protocols rely heavily on the efficiency and security of important cryptographic primitives, such as oblivious transfer (OT) for Boolean-based protocols and oblivious linear evaluation (OLE) for arithmetic-based protocols. However, a fundamental limitation of these primitives is their reliance on public cryptography, as proven by Impagliazzo and Rudich \cite{IR99}. This dependence on public cryptography is a significant drawback in terms of both performance and security, as it requires more computational resources and is vulnerable to attacks from quantum computers, as demonstrated by Shor's algorithm \cite{Sho95}. To address these issues and ensure the safe deployment of SMC methods in the face of quantum computing, it is crucial to develop SMC methods that are secure against quantum attacks while maintaining state-of-the-art performance levels.


\section*{A quantum era}

The second quantum revolution is upon us and quantum technology has advanced to a point where we can integrate its unique features into complex engineering systems. Quantum cryptography, in particular, has been a major area of focus, with research aiming to develop protocols that offer advantages over their classical counterparts. As outlined in \cite{B15, PSAN13}, these advantages can take two forms:

\begin{enumerate}
\item Improving security requirements and achieving information-theoretically secure protocols or those that require fewer computational assumptions;
\item Developing new cryptographic primitives that were previously unattainable using classical techniques.
\end{enumerate}
While quantum key distribution (QKD) is the most well-known application of quantum cryptography, other cryptographic tasks such as bit commitment \cite{CK11}, coin flipping \cite{CK09}, delegated quantum computation \cite{BFK09}, position verification \cite{Unr14}, and password-based identification \cite{DFSS14, DFLSS09} also play important roles in this field.

Also, the intrinsic randomness provided by quantum phenomena can be leveraged to develop quantum communication protocols for oblivious transfer (OT) \cite{BBCS92}. Importantly, there is a significant difference between classical and quantum OT from a security standpoint, as the latter can be achieved with only the assumption of the existence of quantum-hard one-way functions \cite{GLSV21, BCKM21}. This means quantum OT can be based solely on symmetric cryptography, requiring weaker security assumptions than classical OT. Furthermore, these quantum protocols often possess the desirable property of everlasting security, which guarantees information-theoretic security after the execution of the protocol. This greatly improves the security of SMC protocols, allowing them to rely solely on symmetric cryptography and possess the important feature of everlasting security. With regards to the oblivious linear evaluation (OLE) primitive, it is known that it can be reduced to OT \cite{KOS16} through classical methods that do not require additional assumptions. Therefore, it seems natural to use quantum OT to generate quantum-secure OLE instances.


\section*{Contributions} %Section - 1.2

Despite the many advances, the adoption of quantum cryptography in secure multiparty computation (SMC) systems has been limited due to the efficiency challenges posed by quantum technology and the need for high throughput of both OT and OLE primitives in boolean- and arithmetic-based SMC.

The aim of this dissertation is to further the adoption of quantum cryptography in SMC systems through three key contributions. The first contribution involves a study comparing the efficiency of classical and quantum protocols. The second contribution is the implementation of a specialized SMC system for genomics analysis utilizing quantum OT. And the final contribution is the development of the first quantum OLE protocol that does not rely on OT. Additionally, we have also created a comprehensive review dedicated solely to quantum OT protocols, which is often overlooked in broader surveys on the topic of ``quantum cryptography''.

We describe the contributions in a bit more detail.

\

\noindent\textbf{Efficiency of classical and quantum OT protocols.} As far as we are aware, there is no comparative study on the efficiency of quantum and classical approaches. This is largely due to two factors. From a theoretical perspective, the use of different types of information (quantum and classical) makes it challenging to establish a fair comparison based on the complexity of the protocols. From a practical perspective, there is also a significant gap in the technological maturity between quantum and classical techniques. Quantum technology is still in its early stages, whereas classical processors and communication have undergone decades of development.

%%% HERE : improving english

We compare the complexity and efficiency of classical and quantum protocols, despite their constraints. Both types of protocols can be broken down into two phases: precomputation and transfer. The precomputation phase is independent of the parties' inputs and is used to generate the resources needed in the transfer phase, which takes into account the parties' inputs. This phase may not be as efficient as the transfer phase, so for comparison purposes, we focus on the transfer phase. Notably, the transfer phase of quantum OT only involves classical communication, making it possible and fair to compare it to the transfer phase of classical protocols.


We compare the complexity of the transfer phase of two classical OT extension protocols \cite{ALSZ13, KOS15} and an optimized quantum OT protocol in detail. Our conclusion is that the transfer phase of quantum OT is on par with its classical counterparts and has the potential to be more efficient.

\

\noindent\textbf{Quantum assisted secure multiparty computation.} In light of individuals' privacy concerns and legal regulations, it is crucial to handle and study genomic data using highly secure privacy-preserving techniques. We propose a practical secure multiparty computation (SMC) system that utilizes quantum cryptographic protocols to compute a phylogenetic tree from a set of private genome sequences. This system applies several distance-based methods, such as Unweighted Pair Group Method with Arithmetic mean, Neighbour-Joining, and Fitch-Margoliash, in a private setting where the sequences owned by each party are not revealed to other members during the protocol. Instead of using a generic SMC implementation for phylogenetic trees, we develop a specialized private protocol that improves efficiency for this specific use case.


We conduct a theoretical evaluation of the performance and privacy guarantees of our proposed system, providing a complexity analysis and security proof. We also provide a detailed explanation of the implementation details and cryptographic protocols used. We demonstrate the effectiveness and practicality of our quantum-assisted secure phylogenetic tree computation by implementing it using the Libscapi implementation of the Yao protocol, the PHYLIP library and simulated keys of two quantum systems: quantum oblivious key distribution and quantum key distribution. \footnote{ The code can be accessed at the following repo: \href{https://github.com/manel1874/private-phylogenetic-analysis}{github.com/manel1874/private-phylogenetic-analysis}.} We compare our implementation with a classical-only solution and find that both approaches have similar execution times. The only difference between the quantum and classical systems is the time overhead taken by the quantum-assisted approach for oblivious key management.

\

\noindent\textbf{Quantum oblivious linear evaluation protocol.} Our second contribution is a quantum protocol for OLE that provides quantum-UC security in the $\mathcal{F}_{\textbf{COM}}-$hybrid model, which assumes the availability of a commitment functionality, $\mathcal{F}_{\textbf{COM}}$. To ensure security, we leverage the properties of Mutually Unbiased Bases in high-dimensional Hilbert spaces with prime and prime-power dimensions. This approach is motivated by recent theoretical and experimental advancements in quantum cryptography \cite{BPT00,DEBZ10,Zhongetal2015,BHVBFHM18,DHMPPV21} that have opened the door for new solutions in the field.

As far as we are aware, our protocol is the first to propose a quantum-UC secure quantum OLE. Furthermore, it does not rely on any quantum OT implementation, which is a common approach. We design the protocol to handle static corruption adversarial model for both semi-honest and malicious adversaries. Additionally, we introduce a weaker version of OLE, which has potential independent value. We also modify the proposed protocol to generate quantum-UC secure vector OLE (VOLE) and provide bounds on the size of VOLE based on the security parameters.

\section*{Outline}

The results are presented as follows. We start presenting the technical elements required throughout the thesis. Chapter~\ref{chapter_QOT} is devoted to quantum oblivious transfer protocols. Then, in Chapter~\ref{classical-and-quantum-OT} we compare classical and quantum approaches for OT. In Chapter~\ref{ch:phylogenetic-trees}, we presented our implementation of quantum-assisted SMC system applied to phylogeny analysis. In Chapter~\ref{ch:QOLE}, we present our quantum OLE protocol along with its security proof. Finally, in Chapter~\ref{ch:conclusion} we present an overall conclusion of the thesis and propose some future work.


\

\noindent\textbf{Published research}

\noindent This thesis draws on research published in various journals and presents the results of my PhD work, which involved the following projects.

\begin{itemize}
	\item\cite{SMV22} Manuel B. Santos, Paulo Mateus, and Chrysoula Vlachou. “Quantum Universally
Composable Oblivious Linear Evaluation”. 2022. DOI: 10.48550/ARXIV.2204.14171. Poster at QCrypt2022.	
	
	\item\cite{SMP22} Manuel B. Santos, Paulo Mateus, and Armando N. Pinto. “Quantum Oblivious Transfer:
A Short Review”. In: Entropy 24.7 (2022), p. 945.

	\item\cite{SGPM22} Manuel B. Santos et al. “Private Computation of Phylogenetic Trees Based on Quantum
Technologies”. In: IEEE Access 10 (2022), pp. 38065–38088.

	\item\cite{SPM21} Manuel B. Santos, Armando N. Pinto, and Paulo Mateus. “Quantum and classical
oblivious transfer: A comparative analysis”. In: IET Quantum Communication 2.2 (2021), pp. 42–53.

	\item\cite{SGPM21} Manuel B. Santos et al. “Quantum Secure Multiparty Computation of Phylogenetic Trees of SARS-CoV-2 Genome”. In: 2021 Telecoms Conference (ConfTELE). IEEE, Feb. 2021.

    \item\cite{POS+20} Armando N. Pinto et al. “Quantum Enabled Private Recognition of Composite Signals
in Genome and Proteins”. In: 2020 22nd International Conference on Transparent Optical
Networks (ICTON). IEEE, July 2020. 
\end{itemize}

Chapter~\ref{chapter_QOT} is based on \cite{SMP22} and \cite{SGPM22}. Chapter~\ref{classical-and-quantum-OT} is based on the work developed on both \cite{SPM21} and \cite{SGPM22}. Chapter~\ref{ch:phylogenetic-trees} is the combination of \cite{SGPM22, SGPM21, POS+20}. Finally, Chapter~\ref{ch:QOLE} presents all the results from \cite{SMV22}. 

%\

%\

%\noindent\textbf{Further research}

%\noindent Also, during my PhD I was involved in the following projects which are not included in this thesis.

%\begin{itemize}
%	\item \cite{S22} Manuel B. Santos. “PECO: methods to enhance the privacy of DECO protocol”. 2022. In: Cryptology ePrint Archive, Paper 2022/1774.
%	\item \cite{SMV23} Manuel B. Santos, Paulo Mateus, and Chrysoula Vlachou. “Noise-resistent Quantum Oblivious Linear Evaluation”. 2023. DOI: XXXX..
%	\item \cite{TSP23} Miguel Teixeira, Manuel B. Santos, Armando N. Pinto. ``Quantum Blockchain''. 2023. In: IEEE Access.
%	\item \cite{SP23} Manuel B. Santos, Armando N. Pinto. ``Private car crash report''. 2023. In: ICTON.
%\end{itemize}


%\

%\noindent\textbf{Libraries/forks developed}

%\noindent During the course of my PhD, I developed several library forks that were utilized in the publications \cite{SGPM22, SGPM21, TSP23, SP23}.

%\begin{itemize}
%	\item  \href{https://github.com/manel1874/OTKeys}{github.com/manel1874/OTKeys}: an implementation of a string %oblivious transfer based on oblivious keys and random oblivious transfer keys.
%	\item  \href{https://github.com/manel1874/private-phylogenetic-analysis}{github.com/manel1874/private-%phylogenetic-analysis}: an application that securely computes phylogenetic trees based on the PHYLIP package, secure %multiparty computation and quantum technologies.
%	\item \href{https://github.com/manel1874/libscapi}{github.com/manel1874/libscapi}: a fork from Libscapi %\cite{libscapi} that includes modifications to support quantum-secure MPC protocols.
%	\item \href{https://github.com/manel1874/QMP-SPDZ}{github.com/manel1874/QMP-SPDZ}:  a fork from MP-SPDZ %\cite{} that includes modifications to support quantum-secure MPC protocols.
%\end{itemize}


%\bibliography{bibforthesis}
%\bibliographystyle{unsrt}
%\end{document}
