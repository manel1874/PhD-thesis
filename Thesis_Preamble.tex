%%%%%%%%%%%%%%%%%%%%%%%%%%%%%%%%%%%%%%%%%%%%%%%%%%%%%%%%%%%%%%%%%%%%%%%%
%                                                                      %
%     File: Thesis_Preamble.tex                                        %
%     Tex Master: Thesis.tex                                           %
%                                                                      %
%     Author: Andre C. Marta                                           %
%     Last modified : 28 Feb 2014                                      %
%                                                                      %
%%%%%%%%%%%%%%%%%%%%%%%%%%%%%%%%%%%%%%%%%%%%%%%%%%%%%%%%%%%%%%%%%%%%%%%%

% ----------------------------------------------------------------------
% Define document language.
% ----------------------------------------------------------------------

% 'inputenc' package
%
% Accept different input encodings.
% http://www.ctan.org/tex-archive/macros/latex/base/
%
% > allows typing non-english text in LaTeX sources.
%
% ******************************* SELECT *******************************
%\usepackage[latin1]{inputenc} % <<<<< Windows (standard)
%\usepackage[utf8]{inputenc}
%\usepackage[T1]{fontenc}  
%\usepackage[utf8]{inputenc}   % <<<<< Linux
% ******************************* SELECT *******************************


% 'babel' package
%
% Multilingual support for Plain TeX or LaTeX.
% http://www.ctan.org/tex-archive/macros/latex/required//
%
% > sets the variable names according to the language selected
%
% ******************************* SELECT *******************************
%\usepackage{babel} % <<<<< Portuguese
%\usepackage{babel}
%\usepackage[english]{babel} % <<<<< English
% ******************************* SELECT *******************************


% List of LaTeX variable names: \abstractname, \appendixname, \bibname,
%   \chaptername, \contentsname, \listfigurename, \listtablename, ...)
% http://www.tex.ac.uk/cgi-bin/texfaq2html?label=fixnam
%
% Changing the words babel uses (uncomment and redefine as necessary...)
%
\newcommand{\acknowledgments}{Acknowledgments} 
\newcommand{\glossaryname}{List of Abbreviations}
% new LaTeX variable name
%
% > English
%
%\addto\captionsenglish{\renewcommand{\acknowledgments}{Acknowledgements}}
%\addto\captionsenglish{\renewcommand{\listtablename}{List of Tables}}
%\addto\captionsenglish{\renewcommand{\listfigurename}{List of Figures}}
%\addto\captionsenglish{\renewcommand{\nomname}{List of Symbols}}
%\addto\captionsenglish{\renewcommand{\glossaryname}{List of Abbreviations}}
%\addto\captionsenglish{\renewcommand{\acronymname}{List of Acronyms}}
%\addto\captionsenglish{\renewcommand{\bibname}{References}} % Bibliography
%\addto\captionsenglish{\renewcommand{\appendixname}{Appendix}}

% > Portuguese
%
%\addto%\captionsportuguese{\renewcommand{\acknowledgments}{Agradecimentos}}
%\addto\captionsportuguese{\renewcommand{\listtablename}{Lista de Figuras}}
%\addto\captionsportuguese{\renewcommand{\listfigurename}{Lista de Tabelas}}
%\addto\captionsportuguese{\renewcommand{\nomname}{Lista de S\'{i}mbolos}} % Nomenclatura
%\addto\captionsportuguese{\renewcommand{\glossary}{Gloss\'{a}rio}}
%\addto\captionsportuguese{\renewcommand{\acronymname}{Lista de Abrevia\c{c}\~{o}es}}
%\addto\captionsportuguese{\renewcommand{\bibname}{Refer\^{e}ncias}} % Bibliografia
%\addto\captionsportuguese{\renewcommand{\appendixname}{Anexo}} % Apendice


% ----------------------------------------------------------------------
% Define default and cover page fonts.
% ----------------------------------------------------------------------

% Use Arial font as default
%
\renewcommand{\rmdefault}{cmr}
\renewcommand{\sfdefault}{cmr}

% Define cover page fonts
%
%         encoding     family       series      shape
%  \usefont{T1}     {phv}=helvetica  {b}=bold    {n}=normal
%                   {ptm}=times      {m}=normal  {sl}=slanted
%                                                {it}=italic
% see more examples at
% http://julien.coron.free.fr/languages/latex/fonts/
%
\def\FontHb{% 20 pt bold
  \usefont{T1}{cmr}{b}{n}\fontsize{20pt}{20pt}\selectfont}
\def\FontLn{% 16 pt normal
  \usefont{T1}{cmr}{m}{n}\fontsize{16pt}{16pt}\selectfont}
\def\FontLb{% 16 pt bold
  \usefont{T1}{cmr}{b}{n}\fontsize{16pt}{16pt}\selectfont}
\def\FontMn{% 14 pt normal
  \usefont{T1}{cmr}{m}{n}\fontsize{14pt}{14pt}\selectfont}
\def\FontMb{% 14 pt bold
  \usefont{T1}{cmr}{b}{n}\fontsize{14pt}{14pt}\selectfont}
\def\FontSn{% 12 pt normal
  \usefont{T1}{cmr}{m}{n}\fontsize{12pt}{12pt}\selectfont}
\def\FontSb{% 12 pt bold
  \usefont{T1}{cmr}{b}{n}\fontsize{12pt}{12pt}\selectfont}


% ----------------------------------------------------------------------
% Define page margins and line spacing.
% ----------------------------------------------------------------------

% 'geometry' package
%
% Flexible and complete interface to document dimensions.
% http://www.ctan.org/tex-archive/macros/latex/contrib/geometry/
%
% > set the page margins (2.5cm minimum in every side, as per IST rules)
%
\usepackage[nomarginpar]{geometry}	
\geometry{verbose,tmargin=2.5cm,bmargin=2.5cm,lmargin=2.5cm,rmargin=2.5cm}

% 'setspace' package
%
% Set space between lines.
% http://www.ctan.org/tex-archive/macros/latex/contrib/setspace/
%
% > allow setting line spacing (line spacing of 1.5, as per IST rules)
%
\usepackage{setspace}
%\onehalfspacing
%\doublespacing
%\renewcommand{\baselinestretch}{1.4}

% Remove indentation
\setlength{\parindent}{0cm} % Default is 15pt.
%\usepackage{parskip} % double-spacing between paragraphs
%\usepackage{enumitem} % no spacing when using bullets
%\setlist[itemize]{parsep=0pt}
%\setlist[enumerate]{parsep=0pt}

\usepackage{caption} %tables caption
\captionsetup[table]{skip=10pt}
\renewcommand{\arraystretch}{1.5} %table spacing
\usepackage{multirow}
\usepackage{threeparttable} %table notes
%\renewcommand{\TPTnoteSettings}{\small }
%

% ----------------------------------------------------------------------
% Include external packages.
% Note that not all of these packages may be available on all system
% installations. If necessary, include the .sty files locally in
% the <jobname>.tex file directory.
% ----------------------------------------------------------------------

% 'graphicx' package
%
% Enhanced support for graphics.
% http://www.ctan.org/tex-archive/macros/latex/required/graphics/
%
% > extends arguments of the \includegraphics command
%
\usepackage{graphicx}


% 'color' package
%
% Colour control for LaTeX documents.
% http://www.ctan.org/tex-archive/macros/latex/required/graphics/
%
% > defines color macros: \color{<color name>}
%
%\usepackage{color}


% 'amsmath' package
%
% Mathematical enhancements for LaTeX.
% http://www.ctan.org/tex-archive/macros/latex/required/amslatex/
%
% > American Mathematical Society plain Tex macros
%
\usepackage[cmex10]{amsmath}  % AMS mathematical facilities for LaTeX.
\usepackage{amsthm}   % Typesetting theorems (AMS style).
\usepackage{amsfonts} % 


% 'wrapfig' package
%
% Produces figures which text can flow around.
% http://www.ctan.org/tex-archive/macros/latex/contrib/wrapfig/
%
% > wrap figures/tables in text (i.e., Di Vinci style)
%
% \usepackage{wrapfig}


% 'subfigure' package
%
% Deprecated: Figures divided into subfigures.
% http://www.ctan.org/tex-archive/obsolete/macros/latex/contrib/subfigure/
%
% > subcaptions for subfigures
%
%\usepackage{subfigure}
%\usepackage{subfig}
%\captionsetup[subfigure]{labelformat=simple}
%\renewcommand\thesubfigure{(\alph{subfigure})}
\usepackage{subcaption}


% 'subfigmat' package
%
% Automates layout when using the subfigure package.
% http://www.ctan.org/tex-archive/macros/latex/contrib/subfigmat/
%
% > matrices of similar subfigures
%
%\usepackage{subfigmat}


% 'url' package
%
% Verbatim with URL-sensitive line breaks.
% http://www.ctan.org/tex-archive/macros/latex/contrib/url/
%
% > URLs in BibTex
%
% \usepackage{url}


% 'varioref' package
%
% Intelligent page references.
% http://www.ctan.org/tex-archive/macros/latex/required/tools/
%
% > smart page, figure, table and equation referencing
%
%\usepackage{varioref}


% 'dcolumn' package
%
% Align on the decimal point of numbers in tabular columns.
% http://www.ctan.org/tex-archive/macros/latex/required/tools/
%
% > decimal-aligned tabular math columns
%
\usepackage{dcolumn}
\newcolumntype{d}{D{.}{.}{-1}} % column aligned by the point separator '.'
\newcolumntype{e}{D{E}{E}{-1}} % column aligned by the exponent 'E'


% '' package
%
% Reimplementation of and extensions to LaTeX verbatim.
% http://www.ctan.org/tex-archive/macros/latex/required/tools/
%
% > provides the verbatim environment (\begin{verbatim},\end{verbatim})
%   and a comment environment (\begin{comment},  \end{comment})
%
% \usepackage{verbatim}


% 'moreverb' package
%
% Extended verbatim.
% http://www.ctan.org/tex-archive/macros/latex/contrib/moreverb/
%
% > supports tab expansion and line numbering
%
% \usepackage{moreverb}



% 'nomencl' package
%
% Produce lists of symbols as in nomenclature.
% http://www.ctan.org/tex-archive/macros/latex/contrib/nomencl/
%
% The nomencl package makes use of the MakeIndex program
% in order to produce the nomenclature list.
%
% Nomenclature
% 1) On running the file through LATEX, the command \makenomenclature
%    in the preamble instructs it to create/open the nomenclature file
%    <jobname>.nlo corresponding to the LATEX file <jobname>.tex and
%    writes the information from the \nomenclature commands to this file.
% 2) The next step is to invoke MakeIndex in order to produce the
%    <jobname>.nls file. This can be achieved by making use of the
%    command: makeindex <jobname>.nlo -s nomencl.ist -o <jobname>.nls
% 3) The last step is to invoke LATEX on the <jobname>.tex file once
%    more. There, the \printnomenclature in the document will input the
%    <jobname>.nls file and process it according to the given options.
%
% http://www-h.eng.cam.ac.uk/help/tpl/textprocessing/nomencl.pdf
%
% Nomenclature (produces *.nlo *.nls files)
%\usepackage{nomencl}
%\renewcommand{\nomname}{List of Symbols}
%\makenomenclature
%
% Group variables according to their symbol type
%

%%%%%%

%\RequirePackage{ifthen} 
%\ifthenelse{\equal{\languagename}{english}}%
%    { % English
%    \renewcommand{\nomgroup}[1]{%
  %    \ifthenelse{\equal{#1}{R}}{%
 %       \item[\textbf{Roman symbols}]}{%
   %     \ifthenelse{\equal{#1}{G}}{%
      %    \item[\textbf{Greek symbols}]}{%
   %       \ifthenelse{\equal{#1}{S}}{%
    %        \item[\textbf{Subscripts}]}{%
    %        \ifthenelse{\equal{#1}{T}}{%
    %          \item[\textbf{Superscripts}]}{}}}}}%
%    }{% Portuguese
%    \renewcommand{\nomgroup}[1]{%
%      \ifthenelse{\equal{#1}{R}}{%
%        \item[\textbf{Simbolos romanos}]}{%
%        \ifthenelse{\equal{#1}{G}}{%
%          \item[\textbf{Simbolos gregos}]}{%
%          \ifthenelse{\equal{#1}{S}}{%
%            \item[\textbf{Subscritos}]}{%
%            \ifthenelse{\equal{#1}{T}}{%
%              \item[\textbf{Sobrescritos}]}{}}}}}%
%    }%





% 'rotating' package
%
% Rotation tools, including rotated full-page floats.
% http://www.ctan.org/tex-archive/macros/latex/contrib/rotating/
%
% > show wide figures and tables in landscape format:
%   use \begin{sidewaystable} and \begin{sidewaysfigure}
%   instead of 'table' and 'figure', respectively.
%
\usepackage{rotating}

% spacing for inline floats
\setlength{\intextsep}{16pt plus 2.0pt minus 0.0pt} 
% spacing between floats
\setlength{\floatsep}{16pt plus 2.0pt minus 0.0pt}

% 'hyperref' package
%
% Extensive support for hypertext in LaTeX.
% http://www.ctan.org/tex-archive/macros/latex/contrib/hyperref/
%
% > Extends the functionality of all the LATEX cross-referencing
%   commands (including the table of contents, bibliographies etc) to
%   produce \special commands which a driver can turn into hypertext
%   links; Also provides new commands to allow the user to write adhoc
%   hypertext links, including those to external documents and URLs.
%
\usepackage[pdftex]{hyperref} % enhance documents that are to be
                              % output as HTML and PDF
\hypersetup{colorlinks,       % color text of links and anchors,
                              % eliminates borders around links
%            linkcolor=red,    % color for normal internal links
            linkcolor=black,  % color for normal internal links
            anchorcolor=black,% color for anchor text
%            citecolor=green,  % color for bibliographical citations
            citecolor=black,  % color for bibliographical citations
%            filecolor=magenta,% color for URLs which open local files
            filecolor=black,  % color for URLs which open local files
%            menucolor=red,    % color for Acrobat menu items
            menucolor=black,  % color for Acrobat menu items
%            pagecolor=red,    % color for links to other pages
%            pagecolor=black,  % color for links to other pages
%            urlcolor=cyan,    % color for linked URLs
            urlcolor=black,   % color for linked URLs
%	          bookmarks=true,         % create PDF bookmarks
	          bookmarksopen=false,    % don't expand bookmarks
	          bookmarksnumbered=true, % number bookmarks
	          pdftitle={Thesis},
            pdfauthor={Ivo Sousa},
            pdfsubject={Thesis Title},
            pdfkeywords={Thesis Keywords},
            pdfstartview=FitV,
            pdfdisplaydoctitle=true}


% 'glossary' package
%
% Create a glossary.
% http://www.ctan.org/tex-archive/macros/latex/contrib/glossary/
%
%\usepackage{datatool}
% Glossary (produces *.glo *.ist files)
%\usepackage[nonumberlist,style=long,translate=false,nopostdot]{glossaries}
% (remove blank line between groups)
%\setglossary{gloskip={}}
%\renewcommand{\glsnamefont}[1]{\textbf{#1}}
%% (redefine glossary style file)
%\renewcommand{\istfilename}{myGlossaryStyle.ist}
%\setlength{\glsdescwidth}{0.95\linewidth} % left justified
%\makeglossaries


% 'hypcap' package
%
% Adjusting the anchors of captions.
% http://www.ctan.org/tex-archive/macros/latex/contrib/oberdiek/
%
% > fixes the problem with hyperref, that links to floats points
%   below the caption and not at the beginning of the float.
%
\usepackage[figure,table]{hypcap}


% 'natbib' package
%
% Flexible bibliography support.
% http://www.ctan.org/tex-archive/macros/latex/contrib/natbib/
%
% > produce author-year style citations
%
% \citet  and \citep  for textual and parenthetical citations, respectively
% \citet* and \citep* that print the full author list, and not just the abbreviated one
% \citealt is the same as \citet but without parentheses. Similarly, \citealp is \citep without parentheses
% \citeauthor
% \citeyear
% \citeyearpar
%
% ******************************* SELECT *******************************
%\usepackage{natbib}          % <<<<< References in alphabetical list Correia, Silva, ...
\usepackage[numbers,sort&compress]{natbib} % <<<<< References in numbered list [1],[2],...
% ******************************* SELECT *******************************


% ----------------------------------------------------------------------
% Define new commands to assure consistent treatment throughout document
% ----------------------------------------------------------------------

\newcommand{\ud}{\mathrm{d}}                % total derivative
\newcommand{\degree}{\ensuremath{^\circ\,}} % degrees

% Abbreviations

\newcommand{\mcol}{\multicolumn}            % table format

\newcommand{\eqnref}[1]{(\ref{#1})}
\newcommand{\class}[1]{\texttt{#1}}
\newcommand{\package}[1]{\texttt{#1}}
\newcommand{\file}[1]{\texttt{#1}}
\newcommand{\BibTeX}{\textsc{Bib}\TeX}

% Typefaces ( example: {\bf Bold text here} )
%
% > pre-defined
%   \bf % bold face
%   \it % italic
%   \tt % typewriter
%
% > newly defined
%\newcommand{\tr}[1]{{\ensuremath{\textrm{#1}}}}   % text roman
%\newcommand{\tb}[1]{{\ensuremath{\textbf{#1}}}}   % text bold face
%\newcommand{\ti}[1]{{\ensuremath{\textit{#1}}}}   % text italic
%\newcommand{\mc}[1]{{\ensuremath{\mathcal{#1}}}}  % math calygraphy
%\newcommand{\mco}[1]{{\ensuremath{\mathcalold{#1}}}}% math old calygraphy
%\newcommand{\mr}[1]{{\ensuremath{\mathrm{#1}}}}   % math roman
%\newcommand{\mb}[1]{{\ensuremath{\mathbf{#1}}}}   % math bold face
%\newcommand{\bs}[1]{\ensuremath{\boldsymbol{#1}}} % math symbol
%\def\bm#1{\mathchoice                             % math bold
%  {\mbox{\boldmath$\displaystyle#1$}}%
%  {\mbox{\boldmath$#1$}}%
%  {\mbox{\boldmath$\scriptstyle#1$}}%
%  {\mbox{\boldmath$\scriptscriptstyle#1$}}}
%\newcommand{\boldcal}[1]{{\ensuremath{\boldsymbol{\mathcal{#1}}}}}% math bold calygraphy
%\newcommand{\nn}{\nonumber \\}
%\newcommand{\braket}[2]{\langle{#1}|{#2}\rangle}
%\newcommand{\CC}{\mathbb{C}}
%\def\lsim{\mathrel{\rlap{\lower4pt\hbox{$\sim$}}
%    \raise1pt\hbox{$<$}}}                % less than or approx. symbol
%\def\gsim{\mathrel{\rlap{\lower4pt\hbox{$\sim$}}
%    \raise1pt\hbox{$>$}}}                % less than or approx. symbol
%\newcommand{\thm}[1]{\texorpdfstring{\hyperref[thm:#1]{Theorem~\ref*{thm:#1}}}{Theorem~\ref*{thm:#1}}}
%\newcommand{\lem}[1]{\hyperref[lem:#1]{Lemma~\ref*{lem:#1}}}
%\newcommand{\Hamt}{U_H}
%\newcommand{\Hamseg}{V}
%\newcommand{\appseg}{\tilde V}
%\newcommand{\oaa}{V_{\rm oaa}}
%\newcommand{\err}{\Delta}
%\newcommand{\corr}{V_C}
%\newcommand{\corrp}{V^+_C}
%\newcommand{\appcor}{\tilde V_C}
%\newcommand{\ser}{W}
%\newcommand{\segs}{r}
%\newcommand{\obv}{OAA}
%\newcommand{\vd}{V_\Delta}
%\newcommand{\sa}{s_A}
%\newcommand{\appu}{\tilde U}
%\newcommand{\ketup}{\ket{\uparrow}}
%\newcommand{\ketdown}{\ket{\downarrow}}
%\newcommand{\braup}{\bra{\uparrow}}
%\newcommand{\bradown}{\bra{\downarrow}}

\newcommand{\needsrefs}{{\color{red}[find references]}}
\newcommand{\notes}[1]{{\color{blue} #1}}
%\def \lket {|}
%\def \rket {\rangle}
%\def \lbra {\langle}
%\def \rbra {|}


\usepackage{arydshln} %dashed lines

\makeatletter %spacing for the index and page numbers
\renewcommand{\@pnumwidth}{1.7em} 
\renewcommand{\@tocrmarg}{2.55em}
\makeatother

%\usepackage{cite}

\usepackage{epstopdf}		%eps figures
\epstopdfsetup{suffix=}
\DeclareGraphicsRule{.eps}{pdf}{.pdf}{`epstopdf --gsopt=-dPDFSETTINGS=/prepress #1}

\usepackage{color}
%\usepackage{braket}

%\newtheorem{Lemma}{Lemma}
%\newtheorem{Corollary}{Corollary}
%\newcommand{\proof}{\noindent {\bf Proof: }}
%\newcommand{\qed}{$\Box$}

%\usepackage{bbold}
\usepackage{rotating}
\usepackage{bbm}
\usepackage{slashed}


%%%%%%%%%%%%%%%%%%%%%%
%Paper 1 packages & misc%

\usepackage{bm}

\newcommand{\lan}{\langle}
\newcommand{\ran}{\rangle}
\newcommand{\up}{\uparrow}
\newcommand{\dn}{\downarrow}
\newcommand{\vk}{{\bf k}}
\newcommand{\bfk}{{\bf k}}
\newcommand{\vkp}{{\bf k\,'}}
\newcommand{\vq}{{\bf q}}
\newcommand{\vqp}{{\b, f q\,'}}
\newcommand{\vp}{{\bf p}}
\newcommand{\vpp}{{\bf p\,'}}


\usepackage{amsthm}
\newtheorem{theorem}{Theorem}
\newtheorem{corollary}{Corollary}
\newtheorem{lemma}{Lemma}
\newtheorem{definition}{Definition}
\newtheorem{proposition}{Proposition}
\newtheorem{remark}{Remark}



%%%%%%%%%%%%%%%%%%%%%%%%

%%%%%%%%%%%%%%%%%%%%%%%%
%Paper 2 packages & misc%




\newcommand{\ket}[1]{| #1 \rangle}
\newcommand{\bra}[1]{\langle #1 |}
\newcommand{\proj}[1]{\ket{#1}\bra{#1}}
\newcommand{\tr}{\text{Tr}}
\newcommand{\q}[1]{\vec{#1}\cdot\vec{\sigma}}















%\linespread{1.3} %1.3 for one and a half spacing, 1.6 for double
%\usepackage{amsmath, amsthm, amssymb, float, graphicx, caption, subcaption, cite, braket, url,color}
%\usepackage{verbatim}
\usepackage{braket}
%\usepackage{graphics}
%\usepackage[pdftex]{epsfig}
%\usepackage{epsfig}
%\usepackage{epstopdf}
%
%%	\usepackage[nohug,heads=vee]{diagrams}
%%	\diagramstyle[labelstyle=\scriptstyle]
%%	\graphicspath{{./Figures/}}
%	\usepackage[margin=2.5cm]{geometry}
%%	\title{Title}
%%	\author{Chrysoula Vlachou}
%%	\date{}
\newcommand{\N}{\mathbb N}
\newcommand{\R}{\mathbb R}
\newcommand{\C}{\mathbb C}
\newcommand{\Hilb}{\mathcal H}
\newcommand{\HRule}{\rule{\linewidth}{0.5mm}}
\newcommand{\mmobh}{\textlatin{M\"ob}(\mathbb{H})}
\newcommand{\areah}{\textlatin{area}_{\mathbb{H}}}
\newcommand{\dth}{d_{\mathbb{H}}}
\newcommand{\tdth}{$d_{\mathbb{H}}$ }
\def\h{\mathbb H}
%\DeclareMathOperator{\tr}{Tr}
\def\I{\hat I}
\def\ds{\displaystyle}
\def\ppmod{\!\!\!\!\!\pmod}
\newcommand{\walkop}{U_{\text{walk}}}
%\newcommand{\proj}[1]{\ket{#1}\bra{#1}}
%\newcommand{\q}[1]{\vec{#1}\cdot\vec{\sigma}}
%\newtheorem{definition}{Definition}
\newtheorem{protocol}{Protocol}
\def\poly{poly}

\def\O{\textbf{\textit{O}}}
\newcommand{\innerproduct}[2]{\langle #1 | #2 \rangle}
\def\mobh{\textlatin{M\"ob}({\mathbb H})}
\def\span{span}