%\documentclass[11pt]{report}



%\begin{document}

\chapter{Technical overview}
%\addcontentsline{toc}{chapter}{Introduction}

\section{Mathematical preliminaries}

Recall, we use the notation $s\leftarrow_{\$}S$ to describe a situation where an element $s$ is drawn uniformly at random from the set $S$.

Throughout this thesis, Alice plays the role of the sender and Bob plays the role of the receiver.

Introduce $\mathcal{O}$ notation. It is used in chapter 4.

%$\bm{v}_J$ for set J, is the "truncation" of $\bm{v}$ to J.

%********************************** %First Section  **************************************
\section{Secure Multiparty Computation}


Estrutura da introdução:

- Cometar que não sabemos mais do que o output da computação. Dar o exemplo da média de pesos. 2 pessoas sabemos o resultado. 3 já não. Ainda assim, pode revelar alguma coisa a mais. Note that, pratically, we can put together other PET such as Differential Privacy in oder to do this.




Talk about two approaches: boolean and arithmetic. Discuss the advantages and disadvantages of each.

\subsection{Boolean approach}

Boolean approach is based on the Yao protocol. In order to do it we need OT. We start by presenting OT and then we describe the Yao protocol.


\subsubsection{Oblivious Transfer}

The study of oblivious transfer (OT) has been very active since its first proposal in 1981 by Rabin \cite{Rabin81}. The importance of OT comes from its wide number of applications. More specifically, one can prove that OT is equivalent to the secure two-party computation of general functions \cite{Y86, K88}, i.e. one can implement a secure two-party computation using OT as its building block. Additionally, this primitive can also be used for secure multi-party computation (SMC) \cite{KOS16}, private information retrieval \cite{Che04}, private set intersection \cite{MEP17}, and privacy-preserving location-based services \cite{BHM+19}. 









Definition:

use the concealing property and the obliviousness property (used in chapter 4)

Small classical review 

Base OT vs Extended OT

\subsubsection{Yao protocol}\label{yaoProtocol}

Description

Optimizations

Security

Generalizations of Yao: GMW, BMR

\subsection{Arithmetic approach}

\subsubsection{Oblivious Linear Evaluation}

\subsubsection{SPDZ}





%********************************** %Second Section  **************************************
\section{Quantum Information}

$\mathcal{B}(\mathcal{H})$ is the set of positive semi-definite operators with unitary trace acting on an Hilbert space $\mathcal{H}$. {\cv It is used in chapter 3.2.5 Noisy-quantum-storage model}

\subsection{Quantum states representation}

\subsection{Entropy}

\subsection{Two-universal functions}

\subsection{Mutually Unbiased Basis}




%********************************** %Third Section  **************************************
\section{Universal Composability}


%********************************** %Fourth Section  **************************************
\section{Functionality definitions}


%\bibliography{bibforthesis}
%\bibliographystyle{unsrt}
%\end{document}
