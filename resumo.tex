

\chapter*{Resumo}

\addcontentsline{toc}{section}{Resumo}
A criptografia quântica é o campo da criptografia que explora as propriedades quânticas da matéria. Geralmente, visa desenvolver primitivas fora do alcance da criptografia clássica e melhorar as implementações clássicas existentes. Embora grande parte do trabalho neste campo se foque na distribuição de chaves quânticas (\textit{quantum key distribution}, QKD), também têm existido desenvolvimentos cruciais para a compreensão e desenvolvimento de outras primitivas criptográficas, como a transferência oblívia quântica (\textit{quantum oblivious transfer}, QOT). Pode-se mostrar a semelhança entre a estrutura de aplicação das primitivas QKD e QOT. Assim como os protocolos QKD permitem comunicação com segurança quântica, os protocolos QOT permitem computação com segurança quântica. No entanto, as condições sob as quais o QOT é totalmente seguro quântico têm sido sujeitas a um intenso estudo. Nesta tese, começamos por fazer um levantamento do trabalho desenvolvido em torno do conceito de OT dentro da criptografia quântica teórica. Aqui concentramo-nos em alguns protocolos propostos e nos seus requisitos de segurança. Revisitamos os resultados de impossibilidade que intimidam esta primitiva e discutimos vários modelos quânticos de segurança sob os quais é possível provar a segurança do QOT.

A aplicação mais famosa do OT está no domínio da computação multipartidária segura (\textit{secure multiparty computation}, SMC). Esta tecnologia tem o potencial de ser disruptiva nas áreas de análise e computação de dados. Esta permite que vários participantes calculem um certa função, preservando a privacidade dos seus dados. No entanto, a maior parte da segurança e eficiência do protocolos SMD dependem da segurança e eficiência do OT. Por esta razão, fazemos uma comparação detalhada entre a complexidade da QOT baseada em chaves oblívias e dois dos protocolos OT clássicos mais rápidos.

Seguindo a comparação teórica entre OT quântico e clássico, integramos e comparamos ambas as abordagens dentro de um sistema SMC baseado em sequências genéticas. Em resumo, propomos um sistema SMC auxiliado por protocolos criptográficos quânticos com o objectivo de computar uma árvore filogenética a partir de um conjunto de sequências genéticas privadas. Este sistema melhora significativamente a privacidade e a segurança da computação graças a três protocolos criptográficos quânticos que fornecem segurança aprimorada contra ataques de computadores quânticos. Este sistema adapta vários métodos baseados em distância (Unweighted Pair Group Method with Arithmetic mean, Neighbour-Joining, Fitch-Margoliash) num ambiente privado onde as sequências de cada participante não são divulgadas aos demais membros presentes no protocolo. Avaliamos teoricamente as garantias de desempenho e privacidade do sistema através de uma análise de complexidade e prova de segurança, e fornecemos uma extensa explicação dos detalhes de implementação e protocolos criptográficos. Implementamos este sistema com base na implementação Libscapi do protocolo de Yao, na biblioteca PHYLIP e em chaves simuladas de dois sistemas quânticos: distribuição de chaves oblívias quânticas e distribuição de chaves quânticas. Comparamos esta implementação com uma solução somente clássica e concluímos que ambas as abordagens apresentam tempos de execução semelhantes. A única diferença entre os dois sistemas é a sobrecarga de tempo tomada pelo sistema de gestão de chaves oblívias da abordagem quântica.

Finalmente, apresentamos o primeiro protocolo quântico de  avaliação linear oblívia (\textit{oblivious linear evaluation}, OLE). O OLE é uma generalização do OT, em que dois participantes calculam de forma oblívia uma função linear, $f(x) = ax + b$. Ou seja, cada participante fornece os seus dados de forma privada, a fim de calcular o resultado $f(x)$ que se torna conhecido por apenas um deles. Do ponto de vista estrutural e de segurança, o OLE é fundamental para protocolos SMC baseados em circuitos aritméticos. No caso clássico, sabe-se que o OLE pode ser gerado com base no OT, e as contrapartes quânticas desses protocolos podem, em princípio, ser construídas como extensões directas baseadas em QOT. Aqui, apresentamos o primeiro, protocolo quântico OLE que, além disso, não depende de QOT. Começamos apresentando um protocolo semi-honesto e depois estendemo-lo para o cenário desonesto através de uma estratégia \textit{commit-and-open}. O nosso protocolo usa estados quânticos para calcular a função linear, $f(x)$, em campos de Galois de dimensão prima, $GF(d) \cong \mathbb{Z}_d$, ou dimensão de potência prima, $GF(d^M)$. Estas construções utilizam a existência de um conjunto completo de \textit{mutually unbiased bases} em espaços de Hilbert de dimensão de potência prima e o seu comportamento linear sobre os operadores de Heisenberg-Weyl. Também generalizamos o nosso protocolo para obter uma versão vectorial do OLE, onde são geradas várias instâncias de OLE, tornando o protocolo mais eficiente. Provamos que os protocolos têm segurança estática no âmbito da composição universal quântica.

\vfill
\begin{flushleft}
\textbf{Palavras-chave:} criptografia qu\^{a}ntica, passeios qu\^{a}nticos, mem\'{o}rias qu\^{a}nticas, transi\c{c}\~{o}es de fase topol\'{o}gicas, estados de fronteira
\end{flushleft}
