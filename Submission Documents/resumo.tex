\documentclass[12pt]{report}
\linespread{1.3}
\usepackage[margin=2.5cm]{geometry}
\begin{document}
	


\noindent \textbf{T\'{i}tulo:} Passeios qu\^{a}nticos em criptografia e transi\c{c}\~{o}es de fase topol\'{o}gicas a temperature finita\\
\textbf{Nome:} Chrysoula Vlachou\\
\textbf{Doutoramento em:} F\'{i}sica\\
\textbf{Orientador:} Paulo Alexandre Carreira Mateus\\
\textbf{Co-Orientador:} Nikola Paunkovi\'c\\
\vspace{2\baselineskip}

\underline{\textbf{Resumo}}\\

\addcontentsline{toc}{section}{Resumo}
A criptografia qu\^{a}ntica \'{e} uma das mais ativas \'{a}reas de investiga\c{c}\~{a}o, uma vez que se mostrou que a segurança de criptosistemas cl\'{a}ssicos atualmente em utiliza\c{c}\~{a}o ficam comprometidos face a advers\'{a}rios com acesso a computadores qu\^{a}nticos. Na primeira parte desta tese, propomos novos e seguros criptosistemas qu\^{a}nticos baseados em passeios qu\^{a}nticos. Estes \'{u}ltimos t\^{e}m-se vindo a revelar de grande utilidade para diversas tarefas de computa\c{c}\~{a}o qu\^{a}ntica. Em particular, apresentamos um criptosistema qu\^{a}ntico de chave p\'{u}blica seguro, como alternativa aos an\'{a}logos cl\'{a}ssicos, cuja seguran\c{c}a cai por terra face a advers\'{a}rios qu\^{a}nticos. Propomos ainda tr\^{e}s novos protocolos qu\^{a}nticos de distribui\c{c}\~{a}o de chave e analisamos as suas propriedades de seguran\c{c}a e robustez.
A cria\c{c}\~{a}o de mem\'{o}rias qu\^{a}nticas est\'{a}veis de longa dura\c{c}\~{a}o é hoje em dia um dos maiores obst\'{a}culos tecnol\'{o}gicos na constru\c{c}\~{a}o de computadores qu\^{a}nticos escal\'{a}veis, e a sua exist\^{e}ncia ou n\~{a}o tem consequ\^{e}ncias s\'{e}rias em ambas a criptografia cl\'{a}ssica e a criptografia qu\^{a}ntica. Na segunda parte desta tese, estudamos o comportamento, a temperatura finita, de sistemas exibindo ordem topol\'{o}gica, j\'{a} que possuem propriedades \'{u}nicas que podem permitir a constru\c{c}\~{a}o de mem\'{o}rias qu\^{a}nticas. Utilizando a fidelidade entre dois estados qu\^{a}nticos e a condi\c{c}\~{a}o de transporte paralelo de Uhlmann no espa\c{c}o das purifica\c{c}\~{o}es de matrizes densidade, investigamos a exist\^{e}ncia de transi\c{c}\~{o}es de fase topol\'{o}gicas a temperatura finita. Provamos ainda a robustez, em fun\c{c}\~{a}o da temperatura, dos estados de fronteira entre duas fases topol\'{o}gicas distintas. Esta an\'{a}lise mostra que n\~{a}o existem transi\c{c}\~{o}es t\'{e}rmicas de fase e que as propriedades topol\'{o}gicas, presentes a temperatura zero, desaparecem gradualmente \`{a} medida que a temperatura sobe. O nosso estudo dos modos de Majorana (estados de fronteira de supercondutores topol\'{o}gicos), a temperatura baixa mas finita, sugere que podem ser realisticamente utilizados para produzir mem\'{o}rias qu\^{a}nticas.
Aplicamos ainda a mesma an\'{a}lise aos Hamiltonianos efetivos provenientes de protocolos de passeios qu\^{a}nticos que se sabe simularem fases topol\'{o}gicas. Os resultados obtidos condizem com os anteriores, indicando a n\~{a}o exist\^{e}ncia de transi\c{c}\~{o}es t\'{e}rmicas de fase. Para mais, neste caso, a nossa an\'{a}lise revela a exist\^{e}ncia de transi\c{c}\~{o}es param\'{e}tricas de fase a temperatura finita, devido \`{a} periodicidade temporal dos protocolos de passeios qu\^{a}nticos.
Finalmente, investigamos a exist\^{e}ncia de transi\c{c}\~{o}es de fase a temperatura finita em sistemas topol\'{o}gicos fora de equil\'{i}brio. Ainda hoje existem duas formas de inferir a possibilidade de transi\c{c}\~{o}es de fase a temperatura finita para tais sistemas, que levam a resultados contradit\'{o}rios. Derivamos analiticamente as quantidades em causa e identificamos a origem do paradoxo. Discutimos ainda qual dos m\'{e}todos melhor captura a natureza multi-corporal de sistemas topol\'{o}gicos e pode ser utilizado em implementa\c{c}\~{o}es reais.

\vfill
\begin{flushleft}
\textbf{Palavras-chave:} criptografia qu\^{a}ntica, passeios qu\^{a}nticos, mem\'{o}rias qu\^{a}nticas, transi\c{c}\~{o}es de fase topol\'{o}gicas, estados de fronteira
\end{flushleft}
\end{document}