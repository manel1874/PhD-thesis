\documentclass[11pt]{report}
\usepackage[margin=2.5cm]{geometry}
\linespread{1.3}
\begin{document}

\noindent\textbf{T\'itulo:} Passeios qu\^{a}nticos em criptografia e transi\c{c}\~{o}es de fase topol\'{o}gicas a temperatura finita\\

\noindent\underline{\textbf{Resumo alargado em Portugu\^{e}s}}\\
Na primeira parte desta tese, apresentamos o nosso trabalho em criptografia qu\^{a}ntica baseada em passeios  qu\^{a}nticos. Propomos um novo e seguro sistema qu\^{a}ntico de chave p\'{u}blica, no qual a chave p\'{u}blica utilizada para a encripta\c{c}\~{a}o de mensagens \'{e} um estado qu\^{a}ntico gerado por um passeio qu\^{a}ntico. Uma vez que a seguran\c{c}a de muitos dos sistemas de chave p\'{u}blica cl\'{a}ssicos atualmente utilizados fica comprometida face a advers\'{a}rios com acesso a computadores qu\^{a}nticos, cremos que o nosso protocolo oferece uma alternativa \'{u}til para as comunica\~{c}\~{o}es qu\^{a}nticos emergentes. Para mais, a nossa proposta apresenta uma melhoria em rela\c{c}\~{a}o a propostas anteriores, em que a chave p\'{u}blica \'{e} gerada via rota\c{c}\~{o}es de qubits, um de cada vez, em estados separ\'{a}veis. Os estados resultantes de um passeio qu\^{a}ntico s\~{a}o, em geral, estados entrela\c{c}ados, pelo que uma terceira parte que escute a comunica\c{c}\~{a}o necessita, em princ\'{i}pio, de aplicar opera\c{c}\~{o}es substancialmente mais complexas para inferir a chave privada e/ou a mensagem. Consequentemente, a seguran\c{c}a, em termos pr\'{a}ticos, do nosso protocolo \'{e} maior.

Utilizamos ainda passeios qu\^{a}nticos para desenhar e analisar novos protocolos seguros de distribui\c{c}\~{a}o qu\^{a}ntica de chaves. Nessa perspetiva, propomos dois novos protocolos de distribui\c{c}\~{a}o qu\^{a}ntica de chaves, em que as partes trocam chaves geradas a partir dos estados obtidos de certos passeios qu\^{a}nticos para establecer uma chave sim\'{e}trica que poder\~{a}o utilizar para a troca de mensagens cifradas ou para fins de autentica\c{c}\~{a}o. Tamb\'{e}m apresentamos uma varia\c{c}\~{a}o semi-qu\^{a}ntica de distribui\c{c}\~{a}o qu\^{a}ntica de chaves, em que uma das partes pode utilizar exclusivamente opera\c{c}\~{o}es cl\'{a}ssicas. Mostramos que este protocolo, que pode ser considerado mais pr\'{a}tico j\'{a} que necessita de menos hardware qu\^{a}ntico, \'{e} robusto contra advers\'{a}rios que escutem ou manipulem a comunica\c{c}\~{a}o. De facto, se um advers\'{a}rio tentar interferir, ser\'{a} detetado pelas partes leg\'{i}timas, que poder\~{a}o abortar o protocolo. At\'{e} ao momento, a distribui\c{c}\~{a}o qu\^{a}ntica de chaves \'{e} a instância mais pr\'{a}tica e segura de criptografia qu\^{a}ntica, e ser\'{a} portanto de interesse estudar, um futuro, implementa\c{c}\~{o}es reais das nossas propostas te\'{o}ricas. Em particular, poder-se-\'{a} fazer uma an\'{a}lise detalhada das diversas estrat\'{e}gias de ataque que um terceiro possa ter na presen\c{c}a de ru\'{i}do, tal como se poder\'{a} adaptar os ataques e defesas gen\'{e}ricos que apresentamos para o caso de implementa\c{c}\~{o}es espec\'{i}ficas. Nesta linha, mostramos que o nosso protocolo de distribui\c{c}\~{a}o qu\^{a}ntica de chaves unidirecional tolera bastante ru\'{i}do devido \'{a} alta dimens\~{a}o do espa\c{c}o de posi\c{c}\~{o}es, o que vai de encontro a v\'{a}rios estudos recentes. Assim, a aplica\c{c}\~{a}o de passeios qu\^{a}nticos para a distribui\c{c}\~{a}o qu\^{a}ntica de chaves revela-se ser muito promissor para aplica\c{c}\~{o}es pr\'{a}ticas.



Em suma, a contribui\c{c}\~{a}o mais relevante do nosso trabalho \'{e} o facto de introduzirmos, pelo que nos parece ser a primeira vez, a utiliza\c{c}\~{a}o de passeios qu\^{a}nticos em criptografia assim\'{e}trica e distribui\c{c}\~{a}o de chaves, e a prova de que muitas das propriedades dos passeios se traduzem em propriedades significativas de seguran\c{c}a dos protocolos criptogr\'{a}ficos. No seguimento deste trabalho, seria particularmente relevante estudar a possibilidade de criar protocolos criptogr\'{a}ficos diferentes, tais como transfer\^{e}ncia obl\'{i}via e esquemas de compromisso, ou outras primitivas como autentica\c{c}\~{a}o de mensagens e assinaturas digitais, a partir de passeios qu\^{a}nticos.

Na segunda parte da tese, explicamos por que a exist\^{e}ncia ou aus\^{e}ncia de mem\'{o}rias qu\^{a}nticas de longo prazo tem s\'{e}rias implica\c{c}\~{o}es em ambas criptografia cl\'{a}ssica e criptografia qu\^{a}ntica. Assim, estudamos o comportamento, a temperaturas finitas, de sistemas exibindo ordem topol\'{o}gica, que est\~{a}o entre os melhores candidatos para o desenho de mem\'{o}rias qu\^{a}nticas. Estudamos as transi\c{c}\~{o}es de fase de sistemas topol\'{o}gicos a temperaturas finitas recorrendo \'{a} fidelidade, como \'{e} de uso comum, assim como recorrendo a uma quantidade distinta, associada  conex\~{a}o de Uhlmann e a fidelidade via a m\'{e}trica de Bures. Aplicamos esta an\'{a}lise a modelos paradigm\'{a}ticos de insuladores topol\'{o}gicos e supercondutores, e mostramos que as propriedades topol\'{o}gicas presentes a termperatura zero desaparecem gradualmente \'{a} medida que a temperatura aumenta. Analisamos ainda um supercondutor topol\'{o}gico trivial, descrito pela teoria BCS. Contrariamente ao caso do supercondutor topol\'{o}gico, ambas as quantidades indicam a exist\^{e}ncia de transi\c{c}\~{o}es t\'{e}rmicas de fase, j\'{a} que o Hamiltoniano BCS efetivo depende explicitamente da temperatura. Explicamos esta diverg\^{e}ncia comportamental identificando a signific\^{a}ncia das contribui\c{c}\~{o}es t\'{e}rmicas e puramente qu\^{a}nticas para as transi\c{c}\~{o}es de fase. Acreditamos que o nosso estudo, que revela esta diferen\c{c}a e clarifica o porqu\^{e} da sua exist\^{e}ncia, possa ser utilizado para examinar diversas propriedades dos sistemas mencionados em experi\^{e}ncias reais. Confirmamos ainda a aus\^{e}ncia de transi\c{c}\~{o}es t\'{e}rmicas de fase em insuladores topol\'{o}gicos e supercondutores, investigando para isso o comportamento de estados de fronteira.

O estudo dos modos de Majorana (estados de fronteira do supercondutor topol\'{o}gico) a temperatura finita sugere que possam ser realisticamente utilizados para obter mem\'{o}rias qu\^{a}nticas. Assim, uma parte relevante do trabalho futuro ser\'{a} um estudo quantitativo aprofundado robustez destes modos face \'{a} temperatura no m\'{e}todo que propomos.



Fazemos ainda a mesma an\'{a}lise para os Hamiltonianos efetivos resultantes de passeios qu\^{a}nticos espec\'{i}ficos de uma s\'{o} part\'{i}cula, que se veio a saber recentemente, simularem todas as fases topol\'{o}gicas em uma e duas dimens\~{o}es. Em particular, estudamos representantes de duas classes sim\'{e}tricas de insuladores topol\'{o}gicos e chegamos \'{a} mesma conclus\~{a}o: a temperatura efetiva apenas apaga propriedades topol\'{o}gicas exibidas a temperatura zero, sem causar qualquer transi\c{c}\~{a}o t\'{e}rmica de fase. Para mais, a periodicidade temporal dos protocolos de passeios qu\^{a}nticos revela transi\c{c}\~{o}es param\'{e}tricas de fase a temperatura finita, um comportamento emergente em sistemas peri\'{o}dicos. Fazemos o nosso estudo para estados de Boltzmann-Gibbs com uma s\'{o} part\'{i}cula com respeito ao Hamiltoniano efetivo de passeios qu\^{a}nticos de uma s\'{o} part\'{i}cula, e tamb\'{e}m aos seus an\'{a}logos multi-part\'{i}culas, e mostramos que o seu comportamento \'{e} consistente. Assim, a an\'{a}lise de part\'{i}culas \'{u}nicas pode-se vir a revelar uma ferramenta matem\'{a}tica muito \'{u}til no estudo dos sistemas multi-part\'{i}culas correspondentes. Os par\^{a}metros que descrevem passeios qu\^{a}nticos tamb\^{e}m podem ser facilmente controlados em testes experimentais, oferecendo uma plataforma de simula\c{c}\~{a}o para sistemas topologicamente ordenados. Consequentemente, ser\'{a} interessante estudar os efeitos realistas de ru\'{i}do que possam gerar estes passeios com part\'{i}culas \'{u}nicas em estados de Boltzmann-Gibbs, e utilizar a nossa an\'{a}lise para investigar as propriedades topol\'{o}gicas dessas experi\^{e}ncias a temperaturas finitas.



Finalmente, estudamos o comportamento da ordem topol\'{o}gica com respeito \'{a} temperatura para sistemas fora de equil\'{i}brio. O protagonista no estudo das transi\c{c}\~{o}es de fase correspondentes para estados puros \'{e} o eco de Loschmidt, para o qual existem v\'{a}rias propostas de generaliza\c{c}\~{a}o para estados mistos: o eco da fidelidade de Loschmidt e o eco interferom\'{e}trico de Loschmidt. Contudo, estas duas quantidades geram previs\~{o}es paradoxais: o eco da fidelidade de Loschmidt n\~{a}o prev\^{e} transi\c{c}\~{o}es de fase a temperatura finita (o que bate certo com o nosso resultado no caso de transi\c{c}\~{o}es topol\'{o}gicas de fase em sistemas em equil\'{i}brio), enquanto que o eco interferom\'{e}trico de Loschmidt mostra a perman\^{e}ncia de transi\c{c}\~{o}es topol\'{o}gicas de fase a temperaturas finitas. De maneira a clarificar a origem desta incongru\^{e}ncia, derivamos analiticamente a forma das suscetibilidades din\^{a}micas associadas. O eco da fidelidade de Loschmidt quantifica o n\'{i}vel de distin\c{c}\~{a}o entre estados em termos de medi\c{c}\~{o}es de propriedades f\'{i}sicas, induzindo uma m\'{e}trica no espa\c{c}o de estados qu\^{a}nticos, enquanto que o eco interferom\'{e}trico de Loschmidt quantifica os efeitos de canais qu\^{a}nticos que agem sobre um estado, induzindo uma m\'{e}trica de pullback sobre o espa\c{c}o dos unit\'{a}rios. Assim, argumentamos que o eco da fidelidade de Loschmidt e a sua suscetibilidade din\^{a}mica associada s\~{a}o uma melhor medida a utilizar no estudo de sistemas multi-part\'{i}culas, enquanto que o caminho interferom\'{e}trico \'{e} prefer\'{i}vel quando se consideram sistemas qu\^{a}nticos genuinamente microsc\'{o}picos. Para mais, experi\^{e}ncias interferom\'{e}tricas involvem a sobreposi\c{c}\~{a}o coerente de dois estados, o que, no caso de sistemas macrosc\'{o}picos multi-part\'{i}culas, \'{e} experimentalmente imposs\'{i}vel de conretizar com a tecnologia atual.

\vfill
\begin{flushleft}
\textbf{Palavras-chave:} criptografia qu\^{a}ntica, passeios qu\^{a}nticos, mem\'{o}rias qu\^{a}nticas, transi\c{c}\~{o}es de fase topol\'{o}gicas, estados de fronteira
\end{flushleft}


\end{document}