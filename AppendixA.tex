%\documentclass[11pt]{report}
%\linespread{1.3} %1.3 for one and a half spacing, 1.6 for double
%\usepackage{amsmath, amsthm, amssymb, float, graphicx, caption, subcaption, cite, braket, url,color}
%%\usepackage[nohug,heads=vee]{diagrams}
%%\diagramstyle[labelstyle=\scriptstyle]
%%\graphicspath{{./Figures/}}
%\usepackage[margin=2.5cm]{geometry}
%%\title{Title}
%%\author{Chrysoula Vlachou}
%%\date{}
%\newtheorem{lemma}{Lemma}
%\newtheorem{theorem}{Theorem}
%\newtheorem{proposition}{Proposition}
%%\theoremstyle{definition}
%\newtheorem{definition}{Definition}
%\newcommand{\N}{\mathbb N}
%\newcommand{\R}{\mathbb R}
%\newcommand{\C}{\mathbb C}
%\newcommand{\Hilb}{\mathcal H}
%\newcommand{\HRule}{\rule{\linewidth}{0.5mm}}
%\newcommand{\mmobh}{\textlatin{M\"ob}(\mathbb{H})}
%\newcommand{\areah}{\textlatin{area}_{\mathbb{H}}}
%\newcommand{\dth}{d_{\mathbb{H}}}
%\newcommand{\tdth}{$d_{\mathbb{H}}$ }
%\def\h{\mathbb H}
%\DeclareMathOperator{\tr}{Tr}
%\def\I{\hat I}
%\def\ds{\displaystyle}
%\def\ppmod{\!\!\!\!\!\pmod}
%\newcommand{\q}[1]{\vec{#1}\cdot\vec{\sigma}}
%
%
%
%\newcommand{\innerproduct}[2]{\langle #1 | #2 \rangle}
%\def\mobh{\textlatin{M\"ob}({\mathbb H})}
%
%
%
%\begin{document}
\appendix
\chapter*{Appendix A\\Analytic derivation of the closed expressions for the fidelity and $\Delta$}
\addcontentsline{toc}{section}{Appendix A}
As already mentioned, the fidelity between two states $\rho$ and $\rho'$ is given by
\begin{equation}
F(\rho,\rho')=\text{Tr}\sqrt{\sqrt{\rho}\rho'\sqrt{\rho}}.
\end{equation}
We consider unnormalized thermal states $\rho=\exp(-\beta H)$ and $\rho'=\exp(-\beta' H')$. At the end of the calculation one must, of course, normalize the expressions appropriately. We wish to find closed expressions for the fidelity and the quantity $\Delta$ with respect to these thermal states. In order to do that we will proceed by finding $e^C$, such that
\begin{equation}
e^{A}e^{B}e^{A}=e^{C},
\end{equation}
for $A=-\beta H$, $B=-\beta'H'$ and, ultimately, take the square root of the result. The previous equation is equivalent to
\begin{equation}
e^{A}e^{B}=e^{C}e^{-A}.
\label{eq:1}
\end{equation}
The Hamiltonians $H$ and $H'$ are taken to be of the form $\q{h}$, and thus we can write
\begin{equation}
e^{A}=a_0+\q{a}, \nonumber \\
e^{B}=b_0+\q{b}, \nonumber \\
e^{C}=c_0+\q{c},
\end{equation}
where all the coefficients are real, with the following constraints:
\begin{eqnarray}
\label{eq:c1}
\begin{cases}
& 1=\det e^{A}=a_{0}^2-\vec{a}^2,\\
\label{eq:c2}
& 1=\det e^{B}=b_{0}^2-\vec{b}^2,\\
\label{eq:c3}
& 1=\det e^{C}=c_{0}^2-\vec{c}^2,
\end{cases}
\end{eqnarray}
which are equivalent to $\tr A = \tr B = \tr C = 0$, since Pauli matrices are traceless. Let us proceed by expanding the LHS and the RHS of Eq.\eqref{eq:1},
\begin{eqnarray}
& (a_0+\q{a})(b_0+\q{b})=(c_0+\q{c})(a_0-\q{a}) \nonumber \\
& \Leftrightarrow a_0 b_0+ a_0\q{b}+\q{a}b_0+ (\q{a})(\q{b}) = c_0 a_0-c_0\q{a}+\q{c}a_0 -(\q{c})(\q{a})  \nonumber \\
& \Leftrightarrow a_0 b_0 +a_0 \q{b}+\q{a}b_0+ \vec{a}\cdot\vec{b}+i(\vec{a}\times \vec{b})\cdot \vec{\sigma}=c_0 a_0-c_0\q{a}+\q{c}a_0 -\vec{c}\cdot\vec{a}-i(\vec{c}\times \vec{a})\cdot\vec{\sigma}.
\end{eqnarray}
Now, collecting terms in $1$, $\vec{\sigma}$ and $i\vec{\sigma}$, we get a system of linear equations on $c_0$ and $\vec{c}$,
\begin{equation}
\label{Eq:LS1}
\begin{cases}
& a_0 b_0+\vec{a}\cdot\vec{b}- a_0 c_0 +\vec{a}\cdot\vec{c}=0,\\
& a_0\vec{b}+b_0\vec{a}+\vec{a}c_0-a_0\vec{c}=0,\\
& \vec{a}\times \vec{b}-\vec{a}\times \vec{c}=0.
\end{cases}
\end{equation}
The third equation from \eqref{Eq:LS1} can be written as $\vec{a}\times(\vec{b}-\vec{c})=0$, whose solution is given by $\vec{c}=\vec{b}+\lambda\vec{a}$, where $\lambda$ is a real number. This means that the solution depends only on two real parameters: $c_0$ and $\lambda$. Hence, we are left with a simpler system given by,
\begin{equation}
\begin{cases}
& a_0 b_0+\vec{a}\cdot\vec{b}- a_0 c_0 +\vec{a}\cdot(\vec{b}+\lambda\vec{a})=0\\
& a_0\vec{b}+b_0\vec{a}+\vec{a}c_0-a_0(\vec{b}+\lambda \vec{a})=0
\end{cases}.
\end{equation}
Or,
\begin{equation}
\begin{cases}
& a_0 c_0 -\lambda\vec{a}^2=a_0 b_0+2\vec{a}\cdot\vec{b}\\
& (a_0\lambda-c_0)\vec{a}=b_0\vec{a}
\end{cases}.
\end{equation}
In matrix form, the above system of equations can be written as
\begin{equation}
\left[\begin{array}{cc}
a_0 & -\vec{a}^2\\
-1 & a_0	
\end{array}\right]\left[\begin{array}{c}
c_0\\
\lambda	
\end{array}
\right]=\left[\begin{array}{c}
a_0 b_0+2\vec{a}\cdot\vec{b}\\
b_0
\end{array}
\right].
\end{equation}
Inverting the matrix, we get
\begin{eqnarray}
\left[\begin{array}{c}
c_0\\
\lambda	
\end{array}
\right]&=\frac{1}{a_0^2-\vec{a}^2}\left[\begin{array}{cc}
a_0 & \vec{a}^2\\
1 & a_0	
\end{array}\right]\left[\begin{array}{c}
a_0 b_0+2\vec{a}\cdot\vec{b}\\
b_0
\end{array}
\right] \nonumber \\
&=\left[\begin{array}{c}
(2 a_0 ^2-1) b_0+2 a_0\vec{a}\cdot\vec{b}\\
2(a_0 b_0+\vec{a}\cdot\vec{b})
\end{array}
\right],
\end{eqnarray}
where we used the constraints \eqref{eq:c1}. Because of the constraints, $c_0$ and $\lambda$ are not independent, namely,
$e^C=c_0+(\vec{b}+\lambda \vec{a})\cdot\vec{\sigma}$, and we get
\begin{equation}
c_0^2-(\vec{b}+\lambda \vec{a})^2=c_0^2-\vec{b}^2-2\lambda \vec{a}\cdot \vec{b}-\vec{a}^2=1.
\end{equation}
Now we want to make $A=-\beta H/2\equiv-\xi \vec{x}\cdot \vec\sigma/2$ and $B=-\beta 'H'\equiv-\zeta \vec{y}\cdot \vec\sigma$, with $\vec{x}^2=\vec{y}^2=1$ and $\xi$ and $\zeta$ real parameters, meaning,
\begin{equation}
a_0=\cosh(\xi/2) \text{ and } \vec{a}=-\sinh(\xi/2) \vec{x},\\
b_0=\cosh(\zeta) \text{ and } \vec{b}=-\sinh(\zeta) \vec{y}.	
\end{equation}
If we write $C=\rho \vec{z}\cdot \vec \sigma$ (because the product of matrices with determinant $1$ has to have determinant $1$, it has to be of this form),
\begin{eqnarray}
c_0 =\cosh(\rho)& =(2 a_0 ^2-1) b_0+2 a_0\vec{a}\cdot\vec{b}= \nonumber \\
&(2\cosh ^2(\xi/2)-1)\cosh(\zeta)+2\cosh(\xi/2)\sinh(\xi/2)\sinh(\zeta)\vec{x}\cdot\vec{y}= \nonumber \\
&\cosh(\xi)\cosh(\zeta)+\sinh(\xi)\sinh(\zeta)\vec{x}\cdot\vec{y}.
\end{eqnarray}
For all the expressions concerning fidelity, we wish to compute $\text{Tr}(e^{C/2})=2\cosh(\rho/2)$. If we use the formula $\cosh(\rho/2)=\sqrt{(1+\cosh(\rho))/2}$, we obtain,
\begin{equation}
\tr(e^{C/2})=2\sqrt{\frac{(1+\cosh(\xi)\cosh(\zeta)+\sinh(\xi)\sinh(\zeta)\vec{x}\cdot\vec{y})}{2}}.
\end{equation}
Hence, if we let $\xi= \beta E/2$, $\vec{x}=\vec{n}$, $\zeta=\beta' E'/2$ and $\vec{y}=\vec{n}'$, then
\begin{equation}
\tr(\sqrt{e^{-\beta H/2}e^{-\beta' H'}e^{-\beta H/2}})=2\sqrt{\frac{(1+\cosh(\beta E/2 )\cosh(\beta'E'/2)+\sinh(\beta E/2)\sinh(\beta'E'/2)\vec{n}\cdot\vec{n}')}{2}}.
\end{equation}
To be able to compute the fidelities, we will just need the following expression relating the traces of quadratic many-body fermion Hamiltonians (preserving the number operator) and the single-particle sector Hamiltonian obtained by projection:
\begin{equation}
\tr(e^{-\beta \mathcal{H}})=\tr(e^{-\beta \Psi^{\dagger}H\Psi})=\det(I+e^{-\beta H}).
\end{equation}
From the previous results, it is straightforward to derive the following formulae for the fidelities concerning the thermal states considered:\\

\begin{eqnarray}
	F(\rho,\rho')&=\prod_{k\in\mathcal{B}}\frac{\tr(e^{-\mathcal{C}_k/2})}{\tr(e^{-\beta \mathcal{H}_k})\tr(e^{-\beta' \mathcal{H}'_k})}\nonumber\\
	&=\prod_{k\in\mathcal{B}}\frac{\det(I+e^{-C_k/2})}{\det^{1/2}(I+e^{-\beta H_k})\det^{1/2}(I+e^{-\beta' H'_k})}\nonumber\\
	&=\prod_{k\in\mathcal{B}}\frac{2+\sqrt{2\left(1+\cosh(E_k/2T)\cosh(E'_k/2T')+\sinh(E_k/2T)\sinh(E'_k/2T')\vec{n}_k\cdot\vec{n}'_k\right)}}{\sqrt{(2+ 2\cosh (E_k/2T))(2+2\cosh (E'_k/2T'))}},\\
	\nonumber
\end{eqnarray}
		
where the matrix $C_k$ is such that $e^{-C_k}=e^{-\beta H_k/2}e^{-\beta' H'_k}e^{-\beta H_k/2}$ and $\mathcal{C}_k=\Psi_k^{\dagger}C_k\Psi_k$ is the corresponding many-body quadratic operator. \\

To compute $\Delta(\rho,\rho')$ one needs, in addition, $\tr \sqrt{\rho}\sqrt{\rho'}$. This can be done along the lines of what was presented above, hence we shall omit the proof for the sake of briefness and directly provide the result: 
\begin{equation}
\tr \sqrt{\rho}\sqrt{\rho'}=\prod_{k\in \mathcal{B}}\frac{2+2\left(\cosh(E_k/4T)\cosh(E'_k/4T')+\sinh(E_k/4T)\sinh(E'_k/4T')\vec{n}_k\cdot\vec{n}'_k\right)}{\sqrt{(2+ 2\cosh (E_k/2T))(2+2\cosh (E'_k/2T'))}}
\end{equation}
 

%
%
%
%\bibliography{bibforthesis}
%\bibliographystyle{unsrt}
%
%\end{document}